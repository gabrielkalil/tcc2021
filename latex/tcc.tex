\documentclass[
	% -- opções da classe memoir --
	12pt,				% tamanho da fonte
	openright,			% capítulos começam em pág ímpar (insere página vazia caso preciso)
	twoside,			% para impressão em recto e verso. Oposto a oneside
	a4paper,			% tamanho do papel. 
	% -- opções da classe abntex2 --
	%chapter=TITLE,		% títulos de capítulos convertidos em letras maiúsculas
	%section=TITLE,		% títulos de seções convertidos em letras maiúsculas
	%subsection=TITLE,	% títulos de subseções convertidos em letras maiúsculas
	%subsubsection=TITLE,% títulos de subsubseções convertidos em letras maiúsculas
	% -- opções do pacote babel --
	english,			% idioma adicional para hifenização
	french,				% idioma adicional para hifenização
	spanish,			% idioma adicional para hifenização
	brazil				% o último idioma é o principal do documento
	]{abntex2}

% ---
% Pacotes básicos 
% ---
\usepackage{lmodern}			% Usa a fonte Latin Modern			
\usepackage[T1]{fontenc}		% Selecao de codigos de fonte.
\usepackage[utf8]{inputenc}		% Codificacao do documento (conversão automática dos acentos)
\usepackage{indentfirst}		% Indenta o primeiro parágrafo de cada seção.
\usepackage{color}				% Controle das cores
\usepackage{graphicx}			% Inclusão de gráficos
\usepackage{microtype} 			% para melhorias de justificação
% ---
		
% ---
% Pacotes adicionais, usados apenas no âmbito do Modelo Canônico do abnteX2
% ---
\usepackage{lipsum}				% para geração de dummy text
% ---

% ---
% Pacotes de citações
% ---
\usepackage[brazilian,hyperpageref]{backref}	 % Paginas com as citações na bibl
\usepackage[alf]{abntex2cite}	% Citações padrão ABNT

\title{Princípios da Igreja Multiplicadora aplicados na vida de proifssionais em movimento global para abertura de frentes evangelizadores e/ou fortalecimento de igrejas.}

%TODO Pensar em como reduzir o título.
\author{Gabriel Kalil}
\date{ }
  
\begin{document}
  
\maketitle
  
\tableofcontents

\section{Introdução}


\chapter{O surgimento da Igreja e os discípulos em movimento evangelizador}

%Objetivo: Estudar o surgimento natural de igrejas a partir da grande comissão e dos primeiros movimentos de dispersão na Igreja Primitiva.

% O conceito de movimento e espontaniedade é que tem que aparecer aqui. Essa visão é que tem que estar aqui.
% Tem que virar um artigo que igrejas e pessoas podem usar. Se Deus me mandar para Pequim, é uma missão estar lá.
% Christopher Wright (a missão de Deus), diz que Deus é um Deus em movimento (ver pespectivas), como Jesus ele põe o povo em movimento.

\section{O surgimento da Igreja e os discípulos em movimento evangelizador}
 
\subsection{O surgimento da Igreja}
% Dados técnicos sobre o que é a Igreja, revisão de teologia sistemática
%Igreja surge entre os discípulos, 
% surge como comunidade entre os primeiros discípulos
% ===== >>> No meu caso, como vou falar em movimento, precisamos colocar o fato que a igreja [nasce no movimento de discípulos de Jesus.], e neste movimento se concretiza.  Definir Igreja Universal e Igreja Local. Lá na frente, vamos precisar desta definição. É muito importante aparecer que ela nasce em pequenos grupos, no movimento de discípulos, num aspecto de Kononia. (Ver 1.3 do TCC Mariane)

O termo "igreja" tem sua origem no grego (ekklesia), uma palavra comum na época de Jesus. Segundo \cite[317]{zac}, era utilizada no mundo grego como uma assembléia de pessoas, reunidas por motivos relacionados à cidade, sendo eles de origem religiosa, cível ou social, ou seja, uma palavra de uso comum. Entre os judeus oriundos da Diáspora, "ekklesia" que inicialmente se utilizava como sinônimo da palavra "synagoge", logo tomou um sentindo distinto. Enquanto "synagoge" continuou a ser usada para identificar o encontro realizado nas sinagogas, "ekklesia" se tornou a palavra que identificava a comunidade daqueles que Deus havia chamado para a salvação\cite[485]{bavinck}. 

No Novo Testamento, a palavra "ekklesia" é utilizadao em 95\% dos casos diretamente relacionada aos cristãos, ou seja, ao grupo de seguidores de Jesus Cristo. O sentido é de congregação ou assembléia dos seguidores de Cristo, ou, nos termos dos reformadores, "a comunidade dos que creem em Cristo e são santificados nele" \cite[318]{zac}. Ainda segundo \cite[318]{zac}, a igreja é referenciada no Novo Testamento tanto no sentido universal como local. 

\subsubsection{Igreja em sentido universal}

Jesus foi a primeira pessoa a utlizar a palavra igreja no Novo Testamento, aplicando-a ao grupo que o cercava (Mateus 11:18), ou seja, a igreja do Messias, o verdadeiro Israel de Deus \cite[911]{berkhof}. A partir daí, o conceito de igreja é desenvolvido para significar o conjunto de todos os crentes salvos por Jesus Cristo em todos os tempos e lugares, uma entidade de aspecto espiritual \cite[318]{zac}. Jesus é o cabeça deste grupo:
\begin{citacao}“Deus colocou todas as coisas debaixo de seus pés e o designou cabeça de todas as coisas para a igreja, que é o seu corpo, a plenitude daquele que enche todas as coisas, em toda e qualquer circunstância”.(Efésios 1.22,23)
\end{citacao}

A igreja universal abrange o conjunto das pessoas humanas que um dia estarão diante de Deus como a noiva imaculada de Cristo, uma grande assembléia de testemunhas (Hebreus 12:1) desde a criação do mundo \cite[607]{bavinck}. A essa igreja cristo amou e a si mesmo se entregou por ela (Eférios 5:25), "aqueles que quando do momento de sua regeneração (Tito 3,3-6), individualmente colocam a sua fé no Senhor Jesus como seu Salvador (At 16,31)" \cite[319]{zac}


\subsubsection{Igreja em sentido local}

Além do seu sentido universal como entidade espiritual, a igreja também se manifesta como grupo local de pessoas que se encontravam regularmente para edificação, conforme Hebreus 10:25 e Efésios 4:12. Esses grupos locais são formados por pessoas que confessam sua fé em Cristo Jesus e que compartilham de um conjunto de doutrinas cristãs, tendo um ponto de referência geográfico comum para seus encontros. A maioria das epístolas do Novo Testamento e as sete cartas do Apocalipse foram enviadas para igrejas locais, sejam elas grupos únicos de uma cidade ou pequenos grupos reunidos nas casas \cite[320]{zac}.

\subsection{A igreja e os discípulos em movimento evangelizador}

A narrativa do livro de Atos dos Apóstolos, já no início menciona a promessa feita por Jesus sobre o evento que marcaria com capacitação sobrenatural, o início do cumprimento da Grande Comissão dada em Mateus 28
\begin{citacao}
Jesus se aproximou deles e disse: "Toda a autoridade no céu e na terra me foi dada. Portanto, vão e façam discípulos de todas as nações, batizando-os em nome do Pai, do Filho e do Espírito Santo. Ensinem esses novos discípulos a obedecerem a todas as ordens que eu lhes dei. E lembrem-se disto: estou sempre com vocês, até o fim dos tempos(At 1.18-20)
\end{citacao}

\subsubsection{Evangelismo}

O cumprimento da promessa no dia de Pentecostes e a subsequente formação da primeira comunidade dos que pertenciam ao "Caminho" se confirmava, pois "a cada dia, o Senhor lhes acrescentava aqueles que iam sendo salvos" (At 2.47). John Stott reconhecendo a característica evangelística dessa igreja, afirma que o crescimento diário não foi um evento esporádico. A igreja embora ainda não organizasse programas de missões, na mesma medida em que adoravam ao Senhor diariamente (At 2.46), também exercitava o seu testemunho. Tanto a adoração quanto a proclamação fluem naturalmente de corações cheios do Espírito Santo \cite[118-119]{stott}. 

O Espírito Santo é um Espírito missionário que deu início a uma igreja evangelizadora. O testemunho diário dessa comunidade foi um veículo para que a igreja crescesse \cite[78]{stott} sob a fundação apostólica. Observamos essa característica expressa também após Pedro ser instrumento de cura milagrosa de um aleijado, provocando ameaças do conselho de líderes. Assim que foram libertos, Pedro e João retornam para o lugar onde estavam os outros irmãos que ao se identificarem com Pedro e João, juntos levantaram a voz em clamor a Deus. Em resposta a esse clamor e capacitados pelo Espírito Santo, continuaram a pregar as boas novas corajosamente (At 4.21-31).

Esses apóstolos, aprendizes de Jesus durante três anos, apresentavam a boa notícia de que o Reino de Deus havia chegado em Jesus utilizando o método aprendido do próprio Mestre: "todos se dedicavam de coração ao ensino dos apóstolos, à comunhão, ao partir do pão e à oração" (Atos 2:42). A evangelização e ensino através de relacionamentos construiua uma comunidade integrada. Enquanto se relacionavam, os discípulos faziam outros discípulos que aprendiam o estilo de evangelismo de Jesus: o evangelismo discipulador.

\subsubsection{Relacioanmento}

A característica relacional das primeiras igrejas era capaz de trazer para o mesmo local de comunhão pessoas de origens absolutamente distintas. Povos, culturas, visões políticas e pressupostos religiosos compartilhavam do mesmo pão. Green cita sobre a igreja de Antioquia (a qual iremos abordar em seguida) que:

\begin{citacao}
Era uma igreja tão comprometida com a comunhão que judeus e gentios convertidos à fé quebraram barreiras seculares e comiam à mesma mesa. Era uma igreja em que pessoas como Manaém (um aristocrata), Paulo (um ex-fariseu super-rígido), Barnabé (um levita de Chipre e antigo proprietário de um campo), Lúcio (um judeu helênico de Cirene) e “Simeão, chamado Níger” (muito provavelmente um africano)
podiam trabalhar juntos em uma liderança harmoniosa de crentes. Essa comunhão amorosa não estava limitada a Antioquia. Paulo agradece a Deus pelo amor dos tessalonicenses (1Ts 1.3.) e ora para que seu amor possa transbordar sempre mais em favor de todas as pessoas (1Ts 3.12). Foi o próprio Deus que
implantou essa coesão interna, e nesse sentido Paulo nem precisava mencioná-la (1Ts 4.9ss.).
\end{citacao}\cite[261]{green}

\subsubsection{Espontâneo e em movimento}

A partir do evento da vinda do Espírito Santo narrado em Atos 2, se seguem uma série de eventos que nos apontam para um movimento dos discípulos, que no desenrolar da normalidade dessa nova vida, foram presentados por oportunidades de distribuirem a graça de Deus (1 Pedro 4:10), testemunharem do que viram e ouviram (1 João 1:3) e apresentarem a Jesus e seu Reino. Ir ao templo para fazer orações era parte integrante da vida normal de um judeu piedoso da época. Pedro e João, em um desses momentos narrado em Atos 3:1 se deparam com uma centa também comum: um aleijado que todos os dias era colocado ao lado da porta do templo para pedir esmolas, evento que certamente testemunharam muitas vezes. Nessa situação de rotina, o impulso do Espírito Santo através de dois discípulos sensíveis operou uma cura milagrosa e atraiu uma multidão que deu para Pedro a plataforma para um sermão evangelístico que em poucas palavras já apresentava Jesus como o messias esperado, sua natureza divina e ressurreição:

\begin{citacao}
	Pedro, percebendo o que ocorria, dirigiu-se à multidão. “Povo de Israel, por que ficam surpresos com isso?”, disse ele. “Por que olham para nós como se ivéssemos feito este homem andar por nosso próprio poder ou devoção? Pois foi o Deus de Abraão, de Isaque e de Jacó, o Deus de nossos antepassados, quem glorificou seu Servo Jesus, a quem vocês traíram e rejeitaram diante de Pilatos,
	apesar de ele ter decidido soltá-lo. Vocês rejeitaram o Santo e Justo e, em seu lugar, exigiram que um assassino fosse liberto. Mataram o autor da vida, mas Deus o ressuscitou dos mortos. E nós somos testemunhas desse fato!
\end{citacao}(Atos 3:12-15)

O próprio Jesus, na normalidade da vida de seus discípulos, proporcionou oportunidades para que o a boa nova fosse proclamada. 

Em paralelo a esse movimento, outro já havia sido iniciado no evento da descida do Espírito Santo: pessoas de várias nações testemunharam juntamente com os judeus o evento que marcou a história e deu para Pedro também uma plataforma para evangelização. Dessa vez, ouvidos impressionados de "partos, medos, elamitas, habitantes da Mesopotâmia, da Judeia, da Capadócia, do Ponto, da província da Ásia, da Frígia, da Panfília, do Egito e de regiões da Líbia próximas a Cirene, visitantes de Roma (tanto judeus como convertidos ao judaísmo), cretenses e árabes" receberam e puderam compartilhar a boa nova de salvação na volta para seus locais de origem. A normalidade da vida e a falta de planos não impediram o Espírito Santo de executar a Sua obra. O mesmo Jesus que curava, ensinava e discipulava na normalidade da vida agora também operava na normalidade da vida de seus discípulos.

\section{A primeira comunidade cristã e seu foco em Jesus como Senhor}

Com o crescimento da jovem igreja vieram também crescentes problemas. Ao martírio de Estêvão se seguiu uma grande onda de perseguição que dispersou a igreja de Jerusalém para a Judéia, Samaria e além. (At 8.1).

Do ponto de vista cultural, a grande comissão começou a partir da evangelização dos judeus. Esse quadro começa a se modificar quando alguns discípulos fugindo da perseguição capitaneada pelos líderes judeus, chegaram a Antioquia da Síria e lá começaram a pregar as boas novas a pessoas de fala grega, o que segundo Stott se refere tanto a judeus de fala grega quanto a gentios, ou seja, pessoas de várias origens étnicas que habitavam a cidade na época \cite[184]{stott}.

TODO não necessário, não vai usar isso. Talvez precise usar isso lá no profissional. Teria que justificar essa construção pra explicar isso, talvez a contextualização é um aspecto mais missional. Encarnacional, conceito de encarnação. Aculturamento (ver que vale a pena abrir esse leque)

TODO princípio da Igreja multiplicadora. Oração, formação de líderes, compaixão e graça, etc.. Tem que amarrar tudo. Colocar os 5 princípios na frente do computador. ORAÇÃO, EVANGELIZAÇÃO DISCIPULADORA (INFLUENCIA O 1 CAPÍTULO), plantação de igrejas (mentalidade), formação de líderes, compaixão e graça (essa parte de contextualização vem aqui). 

%No capítulo 1, teria que trazer Atos 2:42 (mais do que Pedro). Tem 4 princípios, Koinonia (viviam em comunhão), Diaconia (compaixão e graça) => lá na frente, o profissional precisa estar atento a quais as necessidasdes daquele povo (empatia, simpatia, como me relaciono nesse povo), generosidade (partilha do pão), didake (ensino) => ensino fundamentado, doutrina dos apóstolos.
TODO posso gastar algumas linhas para colocar o estilo de vida deles, vai dar a base para os princípios de igreja multiplicadora. Didake (evangelização discipuladora, que ensina). Oração (ver doc MARCOS). Koinonia (trabalharia na frente pequenos grupos), Generosidade (também compaixão e graça). Faltando aqui a formação de líderes (dar um toque nisso).

TODO Preparar a terrar para o que vamos falar lá na frente. Senão lá vamos puxar algo novo que nunca foi falado.

Nese contexto multi cultural, inovações na proclamação da boa nova começaram a ser introduzidas: Jesus era apresentado não somente como o Cristo (identificação com o Messias esperado pelos judeus) mas agora também, e com mais ênfase, como "O Senhor". Essa inovação terminológica e conceitual, tinha primariamente um caráter apologético. No mundo helênico de muitos deuses e senhores, apresentava Jesus como o único Deus e único Senhor. Mais que isso, chamava atenção para a soberania de Jesus sobre todas as forças de destino que ameaçavam as pessoas. \cite[170]{green}. Jesus, o Senhor, tinha autoridade sobre todas as forças, inclusive sobre a morte (Rm 8.38).

\section{Antioquia da Síria, a base de lançamento para missões gentílicas.}
%Aqui é isso mesmo, não tem muito o que mexer

A base de lançamento para o projeto missionário entre os gentios foi a cidade de Antioquia da Síria, conforme (Atos 11:19-26,13:1-3), onde Lucas relata pela primeira vez a pregação do Evangelho para pessoas totalmente pagãs, uma mudança decisiva na história da igreja. A inovação foi aceita pela igreja de Jerusalém e confirmada pelo envio de Barnabé. Este fora comissionado para aprovar e comunicar à nova comunidade o entendimento de Jerusalém de que os gentios convertidos não precisavam cumprir os ritos da Lei de Israel, inclusive a circuncisão. A fé e o batismo eram suficientes para identificá-los com Jesus, independentemente de sua origem étnica. \cite[198]{green}.


Antioquia era a capital da província da Síria, uma cidade extremamente cosmopolita. Apesar de ser de fundação grega, seus estimados 500.000 habitantes tinham origens variadas. A uma grande colônia de Judeus, se somavam habitantes oriundos da Pérsia, Índia, China, latinos. Gregos, judeus, orientais e romanos formavam uma combinação que levou o historiador Josephus a considerá-la a terceira cidade do império, atrás somente de Roma e Alexandria \cite[185]{stott}. A cidade apresentava um ambiente interessante para o alcançe dos gentios uma vez que as barreiras entre os judeus e eles eram muito tênues, haviam muitos convertidos ao judaísmo e os judeus tinham direitos de cidadãos. Por ser um dos grandes centros comerciais e uma das maiores cidades do império, desenvolvera relações comerciais por todo o mundo o que facilitava o trânsito de pessoas num contexto multi nacional e multi cultural, um literal ponto de convergência entre o Ocidente e o Oriente. Vários sistems de crença também se encontravam ali, desde o culto a Zeus e outros deuses gregos, Baal (de origem síria), religiões de mistério, entre outros.\cite[166,167]{green}. Com um contexto cultural e religioso tão variado, Antioquia da Síria poderia ser facilmente  confundida com capitais Européias como Londres ou Berlin de nossos tempos.

\section{Fundada por missionários informais}
TODO retirar, não precisa porque já colocamos todos os elementos anteriormente. Até aqui, dar um panorama muito bom que é a igreja movimento. Aqui fechar, mostrando o exemplo de uma igreja que nasce deste movimento.

Falar sobre características interessantes que possam ter a ver com o estudo
(o pessoal foi parar em Antioquia mais o dna missionário estava dentro deles. A cidade e contexto deram oportunidade para a continuidade do trabalho)

TODO guardar uma pérola para depois, outro momento.TALVEZ para o livro, ou algo assim. As conversas começam apertando este calo.

\bibliography{research}

\end{document}