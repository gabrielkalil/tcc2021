\documentclass[
	% -- opções da classe memoir --
	12pt,				% tamanho da fonte
	openright,			% capítulos começam em pág ímpar (insere página vazia caso preciso)
	twoside,			% para impressão em recto e verso. Oposto a oneside
	a4paper,			% tamanho do papel. 
	% -- opções da classe abntex2 --
	%chapter=TITLE,		% títulos de capítulos convertidos em letras maiúsculas
	%section=TITLE,		% títulos de seções convertidos em letras maiúsculas
	%subsection=TITLE,	% títulos de subseções convertidos em letras maiúsculas
	%subsubsection=TITLE,% títulos de subsubseções convertidos em letras maiúsculas
	% -- opções do pacote babel --
	english,			% idioma adicional para hifenização
	french,				% idioma adicional para hifenização
	spanish,			% idioma adicional para hifenização
	brazil				% o último idioma é o principal do documento
	]{abntex2}

% ---
% Pacotes básicos 
% ---
\usepackage{lmodern}			% Usa a fonte Latin Modern			
\usepackage[T1]{fontenc}		% Selecao de codigos de fonte.
\usepackage[utf8]{inputenc}		% Codificacao do documento (conversão automática dos acentos)
\usepackage{indentfirst}		% Indenta o primeiro parágrafo de cada seção.
\usepackage{color}				% Controle das cores
\usepackage{graphicx}			% Inclusão de gráficos
\usepackage{microtype} 			% para melhorias de justificação

% ---
		
% ---
% Pacotes adicionais, usados apenas no âmbito do Modelo Canônico do abnteX2
% ---
\usepackage{lipsum}				% para geração de dummy text
% ---

% ---
% Pacotes de citações
% ---
\usepackage[brazilian,hyperpageref]{backref}	 % Paginas com as citações na bibl
\usepackage[alf]{abntex2cite}	% Citações padrão ABNT

\title{Princípios da Igreja Multiplicadora aplicados na vida de proifssionais em movimento global para abertura de frentes evangelizadores e/ou fortalecimento de igrejas.}

%TODO Pensar em como reduzir o título.
\author{Gabriel Kalil}
\date{ }

\begin{document}

\maketitle
  
\tableofcontents

\chapter{O surgimento da Igreja e os discípulos em movimento evangelizador}

\section{O surgimento da Igreja e os discípulos em movimento evangelizador}
 
\subsection{O surgimento da Igreja}

O termo "igreja" tem sua origem no grego (ekklesia), uma palavra comum na época de Jesus. Segundo \cite[317]{zac}, era utilizada no mundo grego como uma assembléia de pessoas, reunidas por motivos relacionados à cidade, sendo eles de origem religiosa, cível ou social, ou seja, uma palavra de uso comum. Entre os judeus oriundos da Diáspora, "ekklesia" que inicialmente se utilizava como sinônimo da palavra "synagoge", logo tomou um sentindo distinto. Enquanto "synagoge" continuou a ser usada para identificar o encontro realizado nas sinagogas, "ekklesia" se tornou a palavra que identificava a comunidade daqueles que Deus havia chamado para a salvação\cite[485]{bavinck}. 

No Novo Testamento, a palavra "ekklesia" é utilizadao em 95\% dos casos diretamente relacionada aos cristãos, ou seja, ao grupo de seguidores de Jesus Cristo. O sentido é de congregação ou assembléia dos seguidores de Cristo, ou, nos termos dos reformadores, "a comunidade dos que creem em Cristo e são santificados nele" \cite[318]{zac}. Ainda segundo \cite[318]{zac}, a igreja é referenciada no Novo Testamento tanto no sentido universal como local. 

\subsubsection{Igreja em sentido universal}

Jesus foi a primeira pessoa a utlizar a palavra igreja no Novo Testamento, aplicando-a ao grupo que o cercava (Mateus 16:18), ou seja, a igreja do Messias, o verdadeiro Israel de Deus \cite[911]{berkhof}. A partir daí, o conceito de igreja é desenvolvido para significar o conjunto de todos os crentes salvos por Jesus Cristo em todos os tempos e lugares, uma entidade de aspecto espiritual \cite[318]{zac}. Jesus é o cabeça deste grupo:
\begin{citacao}“Deus colocou todas as coisas debaixo de seus pés e o designou cabeça de todas as coisas para a igreja, que é o seu corpo, a plenitude daquele que enche todas as coisas, em toda e qualquer circunstância”.(Efésios 1.22,23)
\end{citacao}

A igreja universal abrange o conjunto das pessoas humanas que um dia estarão diante de Deus como a noiva imaculada de Cristo, uma grande assembléia de testemunhas desde a criação do mundo \cite[607]{bavinck}. A essa igreja cristo amou e a si mesmo se entregou por ela (Eférios 5:25), "aqueles que quando do momento de sua regeneração (Tito 3,3-6), individualmente colocam a sua fé no Senhor Jesus como seu Salvador (At 16,31)" \cite[319]{zac}

\subsubsection{Igreja em sentido local}


Além do seu sentido universal como entidade espiritual, a igreja também se manifesta como grupo local de pessoas que se encontravam regularmente para edificação, conforme Hebreus 10:25 e Efésios 4:12. Esses grupos locais são formados por pessoas que confessam sua fé em Cristo Jesus e que compartilham de um conjunto de doutrinas cristãs, tendo um ponto de referência geográfico comum para seus encontros. A maioria das epístolas do Novo Testamento e as sete cartas do Apocalipse foram enviadas para igrejas locais, sejam elas grupos únicos de uma cidade ou pequenos grupos reunidos nas casas \cite[320]{zac}. 

Um exemplo famoso é o pequeno grupo que se torna igreja, iniciada na casa de Aquila e Priscilla, cujos nomes são mencionados diversas vezes no Novo Testamento. Segund wilkins, eles eram pessoas comuns, possivelmente donos do seu próprio negócio. Prováveis fundadores da igreja em Éfeso, são citados em várias cidades e iniciaram igrejas em suas residêndicas tanto em Roma como em Éfeso. Apesar de não existirem registros sobre a fundação da igreja de Roma, pode-se assumir que o casal ajudou a inicia-la, conforme \cite[54]{stetzer}.

\subsection{A igreja e os movimento de evangelização discipuladora}

Finalizando seu ministério como "filho do homem", Jesus outorga poder a seus discípulos e os comissiona para replicar seu modelo e continuar o que ele começou. O evangelista Mateus registra esse evento conhecido como a Grande Comissão: 

\begin{citacao}
Jesus se aproximou deles e disse: "Toda a autoridade no céu e na terra me foi dada. Portanto, vão e façam discípulos de todas as nações, batizando-os em nome do Pai, do Filho e do Espírito Santo. Ensinem esses novos discípulos a obedecerem a todas as ordens que eu lhes dei. E lembrem-se disto: estou sempre com vocês, até o fim dos tempos (Mt 28.18-20)
\end{citacao}

\subsubsection{Aprendizes}

No tempo de Jesus, discípulo (do grego mathetes) era uma palavra conhecida, e literalmente significava aprendiz. A palavra matemática, por exemplo, vem da mesma raíz de mathetes e significa "raciocínio seguido por um comportamento", ou seja, um conhecimento aplicado \cite[15]{gtsm}. Assim também os discípulos na época de Jesus eram mais do que estudantes absorvendo conhecimento intelectual. Ao contrário, aprendiam e desenvolviam o trabalho de seu mestre. \cite[15]{gtsm}. No contexto do Novo Testamento, o mesmo termo é utilizado para designar os discípulos de João Batista (Mt 9.14), dos fariseus (Marcos 2.18), além dos discípulo do próprio Jesus \cite[59]{brandao}. No mundo greco romano, discípulos permeavam a sociedade, desde discípulos de filósofos como Socrates até discípulos de rabinos proeminentes como Shammai e Hillel. Tanto uns como outros eram encharcados pelos conhecimentos de seus mestres enquanto os seguiam \cite[209]{shirley}.

A palavra discípulo, no entanto, foi tento seu significado atualizado com o passar dos anos. Entre a primeira menção da palavra discípulo nos evangelhos (Mat 5.1) e sua última menção em Atos 21.16, muito mudou em seu sentido. Discípulo no primeiro século era um seguidor que aprendia, gradualmente realizava enquanto aprendia e finalmente ensinava nos mesmos moldes. A conexão entre intelecto e prática já existia \cite[105]{wilkins}. Começando com a narrativa de Atos dos Apóstolos e continuando pelo Novo Testamento, a palavra discípulo ganhou uma conotação teológica e específica, designando aqueles que se convertiam ao Evangelho, que compunham a "multidão dos discípulos" (Atos 6.2). Ainda em Atos, discípulos eram aqueles que criam em Jesus, eram salvos e se integravam ao Corpo de Cristo (At 9.26)\cite[59-60]{brandao}. Em Antioquia os discípulos foram chamados de cristãos pela primeira vez. Essa designação, provavelmente atribuída por pessoas de fora da comunidade, tinha um aspecto inicialmente pejorativo, identificando uma nova "seita" entre tantas. No entanto, indicava que uma nova fé emergia distintamente do judaísmo, uma nova comunidade de maioria de crentes gentios e com uma nova composição étnica \cite[90]{wan_diaspora_2011}. A palavra discípulo nesse contexto, portanto, ao mesmo tempo em que inclui o aspecto teológico do novo nascimento e identificação com a pessoa de Jesus e incorporação em seu Corpo, retém também o sentido de aprendizado vivencial, teoria e prática. Discípulos são os salvos por Jesus Cristo que passam a segui-lo imitando a seu Senhor.

\subsubsection{Discipulando como o Mestre}

%TODO Chamada ao estilo de vida ao fazer discípulos e estar em movimento. Pensar em fundamento exegético?

O novo fazer discípulos, comandado por Jesus em Mateus 28 inclui diversos aspectos no processo de imitar o modelo do Mestre. As dimensões de evangelismo (ir), agregação ao Corpo de Cristo dos convertidos (batizar), e o posterior aperfeiçoamento desses discípulos (ensinar) são partículas indissociáveis. Além disso, o mandato para fazer outros discípulos ainda é o mesmo para os discípulos de Jesus de nossos tempos e os que ainda ainda virão. Fazer discípulos não é uma etapa do Discipulado e uma ação subsequente ao "agregar" ou "batizar", mas sim resultado direto do chamado de Jesus ao discipulado \cite[64]{brandao}:
\begin{citacao}“Jesus lhes disse: "Venham! Sigam-me, e eu farei de vocês pescacdores de gente". No mesmo instante, deixaram suas redes e o seguiram."(Marcos 1.17,18)
\end{citacao}

\subsubsection{Evangelização Discipuladora}

O cumprimento da promessa no dia de Pentecostes e a subsequente formação da primeira comunidade dos que pertenciam ao "Caminho" se confirmava, pois "a cada dia, o Senhor lhes acrescentava aqueles que iam sendo salvos" (At 2.47). John Stott reconhecendo a característica evangelística dessa igreja, afirma que o crescimento diário não foi um evento esporádico. A igreja embora ainda não organizasse programas de missões, na mesma medida em que adoravam ao Senhor diariamente (At 2.46), também exercitava o seu testemunho. Tanto a adoração quanto a proclamação fluem naturalmente de corações cheios do Espírito Santo \cite[118-119]{stott}. 

O Espírito Santo é missionário e deu início a uma igreja evangelizadora nos moldes de Jesus. O testemunho diário dessa comunidade foi um veículo para que a igreja crescesse \cite[78]{stott} sob a fundação apostólica, ou seja, os primeiros discípulos. Observamos essa característica logo após o evento de Pentecostes, quando Pedro e João foram instrumentos de uma cura milagrosa de um aleijado. Após serem ameaçados pelo conselho de líderes, foram libertos. Ao retornarem para o lugar de onde havian vindo, os outros irmãos, identificados com a recente experiência pela qual passaram Pedro e João, juntos levantaram a voz em clamor a Deus. Em resposta a esse clamor e capacitados pelo Espírito Santo, continuaram a pregar as boas novas corajosamente (At 4.21-31). Os primeiros discípulos haviam aprendido com o mestre durante três anos e agora manifestam aspectos essenciais do estilo de Jesus: relacionamento (estavam juntos), compaixão (Pedro e João acabavam de voltar da prisão), oração (clamavam juntos), evangelização (pregavam corajosamente). Os primeiros seguidores de Jesus iniciavam então uma comunidade com os mesmos elementos que haviam aprendido de seu mestre: "todos se dedicavam de coração ao ensino dos apóstolos, à comunhão, ao partir do pão e à oração" (Atos 2:42). A evangelização e ensino através de relacionamentos construiu uma comunidade integrada. Enquanto se relacionavam, os discípulos faziam outros discípulos que aprendiam o estilo de evangelismo de Jesus: o evangelismo discipulador.

%TODO Aqui poderia expandir e trabalhar melhor estes princípios (pós cap 2). (4 princípios Atos 2:42) => Formação de líderes também.

% O conceito de movimento e espontaniedade é que tem que aparecer aqui. Essa visão é que tem que estar aqui.
% Tem que virar um artigo que igrejas e pessoas podem usar. Se Deus me mandar para Pequim, é uma missão estar lá.
% Christopher Wright (a missão de Deus), diz que Deus é um Deus em movimento (ver pespectivas), como Jesus ele põe o povo em movimento.

\subsubsection{Espontâneo e em movimento}

A partir do evento da vinda do Espírito Santo narrado em Atos 2, se seguem uma série de eventos que nos apontam para um movimento dos discípulos, que no desenrolar da normalidade dessa nova vida, foram presentados por oportunidades de distribuirem a graça de Deus (1 Pedro 4:10), testemunharem do que viram e ouviram (1 João 1:3) e apresentarem a Jesus e seu Reino. Ir ao templo para fazer orações era parte integrante da vida normal de um judeu piedoso da época. Pedro e João, em um desses momentos narrado em Atos 3:1 se deparam com uma centa também comum: um aleijado que todos os dias era colocado ao lado da porta do templo para pedir esmolas, evento que certamente testemunharam muitas vezes. Nessa situação de rotina, o impulso do Espírito Santo através de dois discípulos sensíveis operou uma cura milagrosa e atraiu uma multidão que deu para Pedro a plataforma para um sermão evangelístico que em poucas palavras já apresentava Jesus como o messias esperado, sua natureza divina e ressurreição:

\begin{citacao}
	Pedro, percebendo o que ocorria, dirigiu-se à multidão. “Povo de Israel, por que ficam surpresos com isso?”, disse ele. “Por que olham para nós como se ivéssemos feito este homem andar por nosso próprio poder ou devoção? Pois foi o Deus de Abraão, de Isaque e de Jacó, o Deus de nossos antepassados, quem glorificou seu Servo Jesus, a quem vocês traíram e rejeitaram diante de Pilatos,
	apesar de ele ter decidido soltá-lo. Vocês rejeitaram o Santo e Justo e, em seu lugar, exigiram que um assassino fosse liberto. Mataram o autor da vida, mas Deus o ressuscitou dos mortos. E nós somos testemunhas desse fato!
\end{citacao}(Atos 3:12-15)

O próprio Jesus, na normalidade da vida de seus discípulos, proporcionou oportunidades para que o a boa nova fosse proclamada. 

Em paralelo a esse movimento, outro já havia sido iniciado no evento da descida do Espírito Santo: pessoas de várias nações testemunharam juntamente com os judeus o evento que marcou a história e deu para Pedro também uma plataforma para evangelização. Dessa vez, ouvidos impressionados de "partos, medos, elamitas, habitantes da Mesopotâmia, da Judeia, da Capadócia, do Ponto, da província da Ásia, da Frígia, da Panfília, do Egito e de regiões da Líbia próximas a Cirene, visitantes de Roma (tanto judeus como convertidos ao judaísmo), cretenses e árabes" receberam e puderam compartilhar a boa nova de salvação na volta para seus locais de origem. A normalidade da vida e a falta de planos não impediram o Espírito Santo de executar a Sua obra. O mesmo Jesus que curava, ensinava e discipulava na normalidade da vida agora também operava na normalidade da vida de seus discípulos.

\section{A primeira comunidade cristã entre os gentios}

O movimento de evangelismo discipulador iniciado por Jesus leva os discípulos a avançarem no estabelecimento de novas comunidades. Do ponto de vista cultural, a grande comissão começou a partir da evangelização dos judeus, mas a ordem de Jesus incluía também os gentios (por todo o mudo). O alcançe dos gentios acontece de várias maneiras. Em especial, quando alguns discípulos fugindo da perseguição capitaneada pelos líderes judeus chegaram a Antioquia da Síria, onde começaram a pregar as boas novas a pessoas de fala grega. A comunidade de Antioquia teve um papel fundamental no cumprimento da grande comissão. É a partir de lá que é lançado o projeto missionário entre os gentios, conforme (Atos 11:19-26,13:1-3). Nesse relato, o evangelista Lucas aponta pela primeira vez a pregação do Evangelho para pessoas totalmente pagãs, uma mudança decisiva na história da igreja. A inovação foi aceita pela igreja de Jerusalém e confirmada pelo envio de Barnabé. Este fora comissionado para aprovar e comunicar à nova comunidade o entendimento de Jerusalém de que os gentios convertidos não precisavam cumprir os ritos da Lei de Israel, inclusive a circuncisão. A fé e o batismo eram suficientes para identificá-los com Jesus, independentemente de sua origem étnica. \cite[198]{green}.

Antioquia era a capital da província da Síria, uma cidade extremamente cosmopolita. Apesar de ser de fundação grega, seus estimados 500.000 habitantes tinham origens variadas. A uma grande colônia de Judeus, se somavam habitantes oriundos da Pérsia, Índia, China, latinos. Gregos, judeus, orientais e romanos formavam uma combinação que levou o historiador Josephus a considerá-la a terceira cidade do império, atrás somente de Roma e Alexandria \cite[185]{stott}. A cidade apresentava um ambiente interessante para o alcançe dos gentios uma vez que as barreiras entre os judeus e eles eram muito tênues, haviam muitos convertidos ao judaísmo e os judeus tinham direitos de cidadãos. Por ser um dos grandes centros comerciais e uma das maiores cidades do império, desenvolvera relações comerciais por todo o mundo o que facilitava o trânsito de pessoas num contexto multi nacional e multi cultural, um literal ponto de convergência entre o Ocidente e o Oriente. Vários sistems de crença também se encontravam ali, desde o culto a Zeus e outros deuses gregos, Baal (de origem síria), religiões de mistério, entre outros.\cite[166,167]{green}. Com um contexto cultural e religioso tão variado, Antioquia da Síria poderia ser facilmente  confundida com capitais Européias como Londres ou Berlin de nossos tempos.

\subsection {Relacional desde o princípio}

A comunidade que nasce em Antioquia da Síria surge colocando em prática os princípios do discipulado de Jesus. O aspecto relacional estava presente de uma maneira muito intensa. A nova comunidade conseguia trazer para o mesmo local de comunhão pessoas de origens absolutamente distintas. Povos, culturas, visões políticas e pressupostos religiosos compartilhavam do mesmo pão. Green fala sobre esse aspecto da igreja de Antioquia:

\begin{citacao}
Era uma igreja tão comprometida com a comunhão que judeus e gentios convertidos à fé quebraram barreiras seculares e comiam à mesma mesa. Era uma igreja em que pessoas como Manaém (um aristocrata), Paulo (um ex-fariseu super-rígido), Barnabé (um levita de Chipre e antigo proprietário de um campo), Lúcio (um judeu helênico de Cirene) e “Simeão, chamado Níger” (muito provavelmente um africano)
podiam trabalhar juntos em uma liderança harmoniosa de crentes. Essa comunhão amorosa não estava limitada a Antioquia. Paulo agradece a Deus pelo amor dos tessalonicenses (1Ts 1.3.) e ora para que seu amor possa transbordar sempre mais em favor de todas as pessoas (1Ts 3.12). Foi o próprio Deus que
implantou essa coesão interna, e nesse sentido Paulo nem precisava mencioná-la (1Ts 4.9ss.).
\end{citacao}\cite[261]{green}

%TODO continue here. Expand

\chapter{Profissionais em Trânsito}

\section {Movimentos Migratórios Modernos}

Movimentos como o que deu origem à igreja de Antioquia continuam a acontecer nos dias de hoje. Vários movimentos migratórios modernos tem um potencial similar de levar a mensagem do Evangelho por toda parte. Nos dias de hoje, esses movimentos são conhecidos como movimentos migratórios. De acordo com \cite{iom2020}, várias definições existem sobre o termo com variações relacionadas a aspectos geográficos, legais, políticos, metodológicos, temporais, entre outros. Para fins de simplificação, escolhemos a definição utilizada pelo Departamento de Estatísticas de Migração Internacional das Nações Unidas, que classifica como migrantes quaisquer pessoas que mudaram seu país de residência. Mais especificamente, migrantes são não refugiados vivendo fora de seu país de origem. 

Dados históricos mostram que os movimentos migratórios estão em grande parte relacionados a nível global com transformações nas áres econômica, social, política e tecnológicas. Conforme os processos de globalização se desenvolvem, essas transformações geram impacto nas famílias, ambientes de trabalho e sociedade. Os movimentos migratórios nem sempre são resultado de processos pacíficos. Nos últimos dois anos especialmente, muitos desses eventos tiveram origens traumáticas. Conflitos (como o da Síria, Yemen, República Democrática do Congo, Sudão do Sul), violência extrema (como o do Povo Rohingya que foi forçado a procurar abrigo em Bangladesh) ou instabilidades políticas ou econômicas (como na Venezuela). As mudanças do meio ambiente também tiveram um impacto na mobilidade humana. Migrações em larga escala desencadeadas por catástrofes climáticas e ambientais ocorreram em diversas partes do mundo em 2018 e 2019 em países como Filipinas, China, Índia, e Estados Unidos da América. \cite[19]{iom2020}.

De acordo com a Organização Internacional de Migração, havia em 2019 272 milhões de migrantes globalmente, o equivalente a 3.5\% da população mundial, sendo que 74\% deles em idade produtiva (de 20-64 anos de idade). O número de migrantes internacionais estimado vem crescendo ao longo das últimas cinco décadas, passando de 2.3\% da população mundial (84.4 milhões de pessoas) em 1970 para 3.5\% em 2019, ou seja, um em cada 30 pessoas. De acordo com o mesmo relatório, o país que mais recebeu migrantes foram os Estados Unidos da América, posição essa que vem sendo mantida desde 1970. Desde lá, o número de migrantes quadruplicou, de menos de 12 milhões de migrantes em 1970 para quase 51 milhões em 2019. A Alemanha segue os EUA na lista dos países que mais recebem imigrantes, com aproximadamente 13.1 milhões em 2019 \cite[21]{iom2020}. Um aspecto importante sobre a distribuição internacional de migrantes é que aproximadamente dois terços tiveram tem sua residência em países de alta renda em 2019, ou seja, países cujo Produto Interno Bruto per Capita maior do que 12.536 dólares americanos \cite{world_bank_country_lending_groups}. Observou-se também que embora migrantes internacionais tenham uma tendência de se deslocar para países de alta renda, suas origens são diversas. Alguns países tem uma proporção de cidadãos vivendo em outros países por razões econômicas, políticas, culturais ou de seguraça, mas que foge dos padrões históricos\cite[45]{iom2020}. Dentre os países que mais enviam migrantes para outros países, se destacam a Índia com 17.5 milhões de migrantes enviados, seguida pelo México (11.8 milhões) e China (10.7 milhões). 

\subsection{Diaspora}

Dentre os movimentos migratórios conhecidos na história, uma categoria é de especial interesse do ponto de vista da propagação do conhecimento de Deus entre os povos: a diáspora. O termo diáspora tem sido cada vez mais usado por antropolólogos, teóricos literários e críticos culturais para descrver as migrações em massa e grandes deslocamentos de pessoas da segunda metade do século vinte, especialmente movimentos de independência de áreas previamente colonizadas, refugiados deixando áreas de conflito e fluxos de migração econômica pós Segunda Guerra Mundial \cite[11]{braziel}. No contexto moderno, o capital se torna global e portanto, termos como globalização e capital global se tornaram relevantes no estudo e definições sobre diáspora. A partir do estabelecimento de acordos de comércio internacional como o NAFTA e GATT e a formação de alianças trans-nacionais como a União Européia, adentramos um período histórico onde empresas multi-nacionais exportam indústrias e postos de trabalho, e tem a possibilidade de movimentar estes postos de trabalho de um país para o outro. As fronteiras entre local e global, entre cidadãos nativos e diáspora são cada vez menos distintas \cite[19]{braziel}. Complementarmente, Michele Reis expande o conceito clássico de diáspora (diáspora do povo Judeu) incluindo as três principais ondas de diáspora que influenciaram o mundo: 

\begin{itemize}
	\item O período clássico dos judeus antigos e a diáspora grega;
	\item O período moderno de escravidão e colonização, dividido em (1) expansão do capital europeu entre 1500-1814, (2) a Revolução Industrial de 1815-1914 e (3) o período entre a Primeira e a Segunda Guerras mundiais;
	\item O período contemporâneo, logo após a Segunda Guerra mundial até os dias de hoje.
  \end{itemize}\cite[41-60]{reis}

\section{Diaspora, uma tendência nos dias atuais}

Os seres humanos sempre se deslocaram espacialmente devido a fatores como guerra, fome, perseguição política ou religiosa. A número de pessoas se deslocando em larga escala, no entanto, se intensificou desde o último século, o que resultou em um grande número de diásporas. Várias diásporas de nível global aconteceram no século XX, como por exemplo a diáspora dos judeus na Europa causada pelo nazismo, a diáspora palestina causada pelo reestabelecimento do estado de Israel em 1948, conflitos políticos na Nicarágua, El Salvador, Guatemala, Honduras, a revolução iraniana de 1979, entre outros \cite[25]{wan_diaspora_2011}.

O século XX viu muitos movimentos de diáspora acontecendo, com crescente frequencia e complexidade. Estima-se que a população de diáspora no ano de 2000 era de 175 milhões de pessoas, deslocando-se geograficamente do norte para o sul, e do leste para o oeste em direção aos países mais ricos do mundo, que embora abriguem 16\% da população mundial, receberam mais de 33\% da população de migrantes \cite[26]{wan_diaspora_2011}.

\section{Diáspora profissional}

A diáspora que acontece nos dias atuais é em grande parte resultado da demanda por profissionais qualificados em países desenvolvidos. Barry Chiswik, escrevendo para o Instituto de Economia do Trabalho em Bohn na Alemanha (IZA), cita a crescente demanda por profissionais qualificados nos países da OECD (Organização para Cooperação Econômica e Desenvolvimento), da qual fazem parte uma grande pate dos países desenvolvidos da Europa, Américas, Ásia e Oceania. Segundo Chiswik, a globalização da economia mundial é um produto de vários elementos, sendo que \begin{citacao} um elemento essencial tem sido o desenvolvimento econômico de vários países menos desenvolvidos no passado. Como resultado, muitos produtos previamente produzido por trabalhadores de fábrica de baixa qualificação nos países da OECD são agora produzidos em países menos desenvolvidos. A globalização da economia mundial alterou a especialização de produção internacional. Isso elevou a demanda por profissionais de alta qualificação nas economias avançadas, mas reduziu a demanda por profissionais de baixa qualificação nas economias cujos postos de trabalhos foram transferidos para outros países. \end{citacao}\cite[3]{chiswick_high_2005}

Por consequência, países como Canada, Austrália, Estados Unidos e Japão tem desenvolvidos vistos baseados em um sistema de pontos que privilegia profissionais qualificados \cite[6]{chiswick_high_2005}. A pesquisa realizada pela IZA sobre a demanda por profissionais altamente qualificados envolvendo parte da Alemanha, França, Reino Unido e Holanda, mostrou que as áreas de Tecnologia de Informação,  Recursos Humanos e Finanças são os setores com maior percentual de migrantes de alta qualificação.\cite[7]{bauer_demand_2004}. Outros setores da indústria também contribuem significativamente com migrantes qualificados, como a indústria química e setores de manufatura \cite[18]{bauer_demand_2004}.

Essa tendência parece ser confirmada pelos dados divulgados pelo Escritório Federal para Migração e Refugiados da Alemanha, o BAMF (Bundesamt für Migration und Flüchlinge) sobre o chamado "Blue Card" Europeu. O EU Blue Card é um visto específico para profissionais qualificados, disponível em todos os países da União Européia.

Segundo dados do BAMF, em 5 anos o número de Blue Cards na Alemanha quase triplicou em 5 anos, entre 2013 e 2018. Em 2018, pouco mais de 50\% desse tipo de visto foi concedido a migrantes da Índia, China, Rússia, Turquia e Brasil. Em relação ao resto da União Européia, a Alemanha distribuiu o maior número de vistos para profissionais qualificados, cerca de 84,5\% de todos os vistos do bloco\cite{bamf}.

\section{Discípulos, profissionais, evangelistas}

Conforme observamos através dos dados apresentados neste capítulo, os movimentos migratórios continuam a acontecer nos tempos modernos, envolvendo um percentual significativo da população mundial que continua em trânsito. Dentre os diversos fatores que motivam movimentos migratórios, destacamos o movimento de profissionais qualificados, em alta demanda em todas as partes, especialmente na Europa e Estados Unidos. Na União Européia, destacamos essa realidade, entre outros, pelo crescente número de "Blue Cards" oferecidos pelos paises do bloco entre os anos de 2013 a 2018. Existe portanto, um grande potencial para um movimento evangelístico e de plantação de igrejas através de discípulos de Jesus com habilidades profissionais específicas. Esses discípulos, aos moldes daqueles que plantaram a igreja de Antioquia da Síria, podem, nos dias de hoje, serem instrumentos para evangelismo e discipulado em locais e contextos (sociais e profissionais) onde iniciativas missionárias organizadas teriam dificuldade de chegar. Discípulos comprometidos com Jesus Cristo podem intencionalmente desenvolver competências profissionais que os habilitem a ser uma força evangelística de impacto global, em obediência ao mandato de Jesus. Os princípios da Igreja Multiplicadora, aplicados a esses profissionais em movimento, podem ser instrumentos para o discipulado de muitas nações e plantação de novas igrejas.

\chapter{Princípios da Igreja Multiplicadora}

O comissionamento de Jesus para uma vida de movimento e fazer novos discípulos dele, Jesus, continua sendo o mesmo que Pedro, Paulo, João e outros discípulos implementaram nas primeiras décadas da era moderna, o que exigiu muita criatividade para entregar uma mensagem de esperança para fora do contexto judeu. Em Antioquia da Síria, a primeira igreja entre os gentios foi iniciada a partir de uma série de inovações por parte dos discípulos que ali chegaram. Michael Green, por exemplo cita o crescente uso do termo "Senhor" para se referir a Jesus, como uma alternativa ao termo "Cristo", que carregava uma forte menção ao Messias judeu. Jesus "Senhor" o posicionava em contraste com outros senhores (deuses) do mundo helênico, como o único "Senhor" \cite[170]{green}. Iniciativas como essa nos mostram o zelo dos discípulos em obedecer ao mestre e implementar os princípios que aprenderam de maneira a alcançar a todos. Da mesma maneira, discípulos profissionais dos dias atuais em trânsito global, precisam adaptar os mesmos princípios aprendidos de Jesus de novas maneiras para fazer discípulos.


\section{Igreja Multiplicadora}

A Igreja Multiplicadora é uma visão nascida na denominação batista brasileira e que busca desenvolver multiplicação intencional de discípulos baseada em princípios bíblicos como oração, evangelização discipuladora, plantação de igrejas, formação de líderes e compaixão e graça \cite{igrejaMultiplicVisaoSite}. Conforme Fabrício Freitas, \begin{citacao}Sermos bênção a partir de onde estamos para alcançar as nações por meio de um grande movimento de multiplicação de discípulos e igrejas é o grande objetivo da visão de Igreja Multiplicadora.\end{citacao}\cite[19]{freitas}.

A visão e movimento da Igreja Multiplicadora nasceu a partir da constatação de que é necessário um retorno ao estilo de vida da igreja primitiva, cuja história é registrada no livro de Atos dos Apóstolos nos primeiros dois capítulos. A chegada do Espírito Santo marcou o rumo das igrejas neotestamentárias cujos princípios é preciso resgatar. O fazer discípulos precisa ser o estilo de vida das igrejas, e mais importante, cada discípulo de Jesus Cristo. Conforme Christopher J. H. Wright, 
\begin{citacao}O povo de Deus é um povo de bênção para as nações. Na verdade, afirma Paulo, essa é a Boa-Nova, o evangelho (Gl 3.8). Abençoar as nações é a missão declarada de Deus. É por essa razão que ele chama seu povo à existência, para ser o veículo da missão do Senhor no mundo histórico das nações.\end{citacao}\cite[98]{wright_missao_2012}.

Abençoar nasçoes através de movimentos de discipulado não é uma idéia nova. Dawson Trotman, por exemplo, fundador do conhecido ministério Navegadores na década de 1930 tinha como objetivo espalhar as Boas-Novas por meio de relacionamentos intencionais, capacitando discípulos para a multiplicação. Tomando como base o cuidado um a um, inspirado em duplas como Paulo e Timóteo,  Barnabé e Paulo, os Navegadores enfatizam a importância de investir tempo com novos crentes, individual e coletivamente (em pequenos grupos). Através desses relacionamentos, vida é gerada a partir de vida, seja na oração, encorajamento, aconselhamento ou ensino mútuo.\cite[21]{freitas} \begin{citacao}"Você me ouviu ensinar verdades confirmadas por muitas testemunhas confiáveis. Agora, ensine-as a pessoas de confiança que possam transmiti-las a outros." 2a Timoteo 2.2\end{citacao}


\subsection{Oração}

Orar foi o primeiro movimento dos discípulos após o recebimento da Grande Comissão\cite[28]{brandao}. A importância da oração como estilo de vida dos discípulos é vista na igreja de Atos dos Apóstolos. Ao retornarem para Jerusalém, os discípulos aguardaram a promessa do Espírito Santo e perseveravam em oração:

\begin{citacao}"Todos eles se reuniam em oração com um só propósito, acompanhados de algumas mulheres e também de Maria, mãe de Jesus, e os irmãos dele." Atos 1:14.\end{citacao}

O resultado foi uma proclamação intensa do evangelho, muitos foram adicionados à jovem comunidade e "Todos se dedicavam de coração ao ensino dos apóstolos, à comunhão, ao partir do pão e à oração." Atos 2:42. 

\begin{citacao}A Igreja Multiplicadora apresenta o princípio Oração, com dois objetivos: demonstrar que a oração é essencial para que tudo isso aconteça e fornecer os meios para fazer da igreja local uma igreja multiplicadora através da oração.\end{citacao}\cite[29]{brandao}.

Moisés falava com Deus face a face, como quem fala a um amigo, um relacionamento intenso de amizade que refletia diretamente em seu ministério. Segundo Brandão, "a manutenção de uma vida de oração sem cessar é a plenitude do relacioanmento entre homem e Deus" \cite[30]{brandao}. Orar é então mais que pedir, pelo contrário, envolve diferentes aspectos:

\begin{itemize}
	\item Reconhecimento: reconhecer quem é Deus e qual é a nossa posição em relação a Ele nos leva a prostrarnos diante do Senhor (Lucas 5.8).
	\item Intimidade: Fomos criados para viver em permanente comunhão com Deus, e portanto precisamos investir o melhor do nosso tempo diáro em conversa com Ele, oração e leitura da Palavra.
	\item Santidade: Nossos pecados fazem separação entre nós e Deus, portanto confissão e arrependimento devem fazer parte da oração. Devemos, ao exemplo do salmista Davi, pedir que o Senhor sonde nosso coração e nos leve ao caminho do arrependimento e justiça.
	\item Humildade: Infelizmente em nossos dias, muitos tem dificulade de ter uma atitude de verdadeira submissão a Deus. Orar é se prostrar diante de Deus em total submissão e reverencia, como um escravo vulnerário de frente a seu Senhor (Lucas 5.8, Filipenses 2.10).
	\item Fé: Precisamos crer que o Senhor ouve as nossas orações e no seu tempo, de acordo com seus propósitos, nos escuta.
	\item Submissão à vontade de Deus: A oração que aprendemos com Jesus nos ensina a clamar para que a vontade do Pai seja realizada, não a nossa. No Getsêmani jesus ora "Pai, se queres, afasta de mim este cálice. Contudo, que seja feita a tua vontade, e não a minha" (Lucas 22:42).
  \end{itemize}\cite[30,31]{brandao}.

  A oração portanto é essencial para o cumprimento da grande comissão. A Bíblia tem vários exemplos de homens e mulheres que viram uma resposta sobrenatural e extraordinário de Deus como resposta a suas orações. Através da oração, somos capacitados, motivados e orientados pelo Espírito Santo, assim como os discípulos em Atos 1:8 pregaram com ousadia e o resultado foram igrejas sendo iniciadas por todas as partes. Preciamos orar para viver de fato a plenitude do Espírito Santo para um testemunho frutífero como o a igreja primitiva. \cite[32]{brandao}.

  Da perspectiva da obra missionária, a oração é estratégica.
  \begin{itemize}
	\item A oração é essencial na evangelização: É preciso investir tempo de oração por conversões pois somente o Espírito Santo pode convencer as pessoas do pecado. Precisamos preparar o solo pagando preço da oração para que nossa evangelização seja eficiente.
	\item A oração nos coloca em um papel de parceria com Deus no processo de libertação dos cativos, "O deus deste mundo cegou a mente dos que não creem, para que não consigam ver a luz das boas-novas, não entendendo esta mensagem a respeito da glória de Cristo, que é a imagem de Deus." (2 Cor 4:4).
	\item Através da intercessão, toda a igreja pode se posicionar estrategiamente na obra missionária. Todo discípulo pode causar impacto evangelístico por meio da oração: "Orem no Espírito em todos os momentos e ocasiões. Permaneçam atentos e sejam persistentes em suas orações por todo o povo santo. E orem também por mim. Peçam que Deus me conceda as palavras certas, para que eu possa explicar corajosamente o segredo revelado pelas boas-novas." (Efésios 6:18,19).
	\item Jesus nos ensinou a orar pelos vocacionados, "Estas foram suas instruções: “A colheita é grande, mas os trabalhadores são poucos. Orem ao Senhor da colheita; peçam que ele envie mais trabalhadores para seus campos.", um grande desafio para a obra missionária em todo o mundo. Multiplicação de igrejas implica em multiplicação de líderes.
	\item A Oração é uma poderosa arma para a batalha espiritual, e elemento indispensável da armadura de Deus: "Pois nós não lutamos contra inimigos de carne e sangue, mas contra governantes e autoridades do mundo invisível, contra grandes poderes neste mundo de trevas e contra espíritos malignos nas esferas celestiais. Portanto, vistam toda a armadura de Deus, para que possam resistir ao inimigo no tempo do mal. Então, depois da batalha, vocês continuarão de pé e firmes. Efésios 6:12".
  \end{itemize}\cite[32-34]{brandao}.

\subsection{Evangelização Discipuladora}

\subsubsection{Evangelização}
Evangelizar, é por essência, anunciar o Evangelho, o primeiro passo para a formação de um discípulo de Jesus. A conotação da palavra no Novo Testamento é entre outros de "anunciar", "testemunhar", "proclamar"\cite[54]{brandao}. Paulo em sua carta aos Romanos, exemplifica: "Pois “todo aquele que invocar o nome do Senhor será salvo”. Mas como poderão invocá-lo se não crerem nele? E como crerão nele se jamais tiverem ouvido a seu respeito? E como ouvirão a seu respeito se ninguém lhes falar? E como alguém falará se não for enviado? Por isso as Escrituras dizem: “Como são belos os pés dos mensageiros que trazem boas-novas!”. Nem todos, porém, aceitam as boas-novas, pois o profeta Isaías disse: “Senhor, quem creu em nossa mensagem?”. Portanto, a fé vem por ouvir, isto é, por ouvir as boas-novas a respeito de Cristo." (Romanos 10:13-17).

A evangelização oferece para as pessoas uma oportunidade de receber o Evangelho de maneira clara, de modo que a mesma tenha a possibilidade de responder positivamente. Devemos evangelizar todas as pessoas, "a tempo e fora de tempo" (2 Timóteo 4:2), ou seja, constante proclamação do Evangelho para pessoas do nosso alcance. Evangelizar nos une a Jesus em em compaixão pelas almas perdidas: "Quando viu as multidões, teve compaixão delas, pois estavam confusas e desamparadas, como ovelhas sem pastor." (Mateus 9:36). Segundo Brandão, a evangelização pode ser portanto definida como: \begin{citacao}A comunicação clara e fiel do Evangelho visando, sob o poder e a dependência do Espírito Santo, levar pessoas ao arrependimento e à fé em Jesus Cristo como Senhor e Salvador, tornando-se seus discípulos.\end{citacao}\cite[57]{brandao}.
	
\subsubsection{Discipulado}

A palavra "discipulado" não ocorre na Bíblia, e sim somente o termo "discípulo". O imperativo da grande comissão de Mateus 28 se refere a "fazer discípulos". Discípulo é aquele que segue o ensino de alguém, que aprende aos pés de um mestre. No Novo Testamento, uma palavra comum, inclusive aplicada aos discípulos de João Batista e Moisés (João 9:28). Jesus em seu ministério também chama discípulos paar o seguirem como mestre, uma relação comum na cultura judaica\cite[59]{brandao}. O significado teológico da palavra discípulo começa a ser empregado em Atos dos Apóstolos para desiginar todos aqueles que criam em Jesuse eram consequentemente adicionados à comunidade de fé, a comunidade dos discípulos. O apóstolo Paulo entendeu essa identificação logo no início de seu ministério, pois "Quando Saulo chegou a Jerusalém, tentou se encontrar com os discípulos, mas todos estavam com medo dele, pois não acreditavam que ele tivesse de fato se tornado discípulo." (Atos 9:26). 

O discipulado como chamado primordial de cada cristão começa com um relacionamento intencional que começa desde a primeira relação evangelística, que começa com o chamado, agregação e aperfeiçoamento do novo discípulo. A esse "processo de fazer discípulos multiplicadores por meio do relacioamento intencional de um discípulo com uma pessoa visando torná-la outro discípulo"\cite[64]{brandao} a Igreja Multiplicadora chama de Relacionamento Discipulador, ou RD. "Enquanto o discipulado é o processo de fazer discípulos, o RD é o meio pelo qual esse processo se desenvolve"\cite[64]{brandao}.

\subsubsection{Evangelização Discipuladora}

Tanto evangelização quanto discipulado se completam e precisam estar debaixo da autoridade da grande comissão para serem plenamento entendidos e usados pela igreja. Segundo Brandão,
\begin{citacao}O cumprimento da Grande Comissão exige a aplicação conjunta dessas duas forças: a transmissão de verdades por intermédio da comunicação do Evangelho e seus desdobramentos, o que compreende a sua proclamação, exposição e ensino, inclusive quanto a suas implicações para a vida cristã, e a transmissão de	vida por meio de um relacionamento.\end{citacao} \cite[66]{brandao}.

Na descrição do esforço missionário do apóstolo Paulo podemos observar as várias dimensões da grande comissão na maneira como foi realizado seu ministério entre os efésios:
\begin{citacao}"Quando chegaram, ele lhes disse: “Vocês sabem que, desde o dia em que pisei na província da Ásia até agora, fiz o trabalho do Senhor humildemente e com muitas lágrimas. Suportei as provações decorrentes das intrigas dos judeus e jamais deixei de dizer a vocês o que precisavam ouvir, seja publicamente, seja em seus lares. Anunciei uma única mensagem tanto para judeus como para gregos: é necessário que se arrependam, se voltem para Deus e tenham fé em nosso Senhor Jesus. (...) Mas minha vida não vale coisa alguma para mim, a menos que eu a use para completar minha carreira e a missão que me foi confiada pelo Senhor Jesus: dar testemunho das boas-novas da graça de Deus. Por isso, declaro hoje que, se alguém se perder, não será por minha culpa, pois não deixei de anunciar tudo que Deus quer que vocês saibam. (...) Portanto, vigiem! Lembrem-se dos três anos que estive com vocês, de como dia e noite nunca deixei de aconselhar com lágrimas cada um de vocês. (...) Fui exemplo constante de como podemos, com trabalho árduo, ajudar os necessitados, lembrando as palavras do Senhor Jesus: ‘Há bênção maior em dar que em receber" (Atos 20:18-21,24,26,27,31,35)\end{citacao}

Evangelização Discipuladora, ao modelo do ministério de Paulo, precisa incluir transmissão de verdades e de vida, ensino e relacionamento, comunicação e convívio, evangelização e relacionamento.

\subsection{Plantação de Igrejas}

A Grande Comissão nos entrega a missão de fazer discípulos. A igreja é uma comunidade onde discípulos são agregados e aperfeiçoados, e a partir dos quais novos discípulos são formados e o ciclo se repete. O RD como ferramenta para a Evangelização Discipuladora agrega discípulos a uma igreja local. Nos lugares onde não há igrejas, esse processo implica na plantação de novas igrejas. Dessa maneira o Evangelho pode ser permanentemente enraizado em uma cidade ou bairro por exemplo, e a partir dali se multiplicar\cite[99,100]{brandao}. Essa foi a estratégia adotada pela igreja primitva, conforme se vê em Atos 14:21-23: \begin{citacao}"Depois de terem anunciado as boas-novas em Derbe e feito muitos discípulos, Paulo e Barnabé voltaram a Listra, Icônio e Antioquia da Pisídia, onde fortaleceram os discípulos. Eles os encorajaram a permanecer na fé, lembrando-os de que é necessário passar por muitos sofrimentos até entrar no reino de Deus. Paulo e Barnabé também escolheram presbíteros em cada igreja e, com orações e jejuns, os entregaram aos cuidados do Senhor, em quem haviam crido."\end{citacao}
Conforme Brandão, \begin{citacao}Tudo indica que já no final do primeiro século a igreja percebeu a necessidade	da ekklesia – igreja local – para o enraizamento do Evangelho nas cidades,	províncias e regiões mais distantes entre os gentios. Isso significa, uma vez mais,	que o ato de evangelizar alcança uma pessoa, mas o processo de plantar igrejas	faz com que o Evangelho permaneça para as futuras gerações.\end{citacao}\cite[101]{brandao}.	

\subsection{Formação de Líderes}
\subsection{Compaixão e Graça}

\chapter{Conclusão}

\bibliography{research}
\end{document}
