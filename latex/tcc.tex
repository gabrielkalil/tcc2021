%%%%%%%%%%%%%%%%%%%%%%%%%%%%%%%%%%%%%%%%%%%% 
%Classe de documento 
%%%%%%%%%%%%%%%%%%%%%%%%%%%%%%%%%%%%%%%%%%%% 


\documentclass[12pt,openright,oneside,a4paper,
english,french,spanish,brazil]{abntex2}

%%%%%%%%%%%%%%%%%%%%%%%%%%%%%%%%%%%%%%%%%%% 
%Pacotes de língua 
%%%%%%%%%%%%%%%%%%%%%%%%%%%%%%%%%%%%%%%%%%% 
\usepackage[utf8]{inputenc} 
\usepackage{indentfirst}

\usepackage{titlesec}
\titleformat{\chapter}{\bfseries}{\thechapter.}{1em}{\MakeUppercase}
\titleformat{\section}{}{\thesection.}{1em}{\MakeUppercase}
\titleformat{\subsection}{\bfseries}{\thesubsection.}{1em}{} 
\titleformat{\subsubsection}{}{\thesubsubsection.}{1em}{}

\usepackage{tocloft}

%%%%%%%%%%%%%%%%%%%%%%%%%%%%%%%%%%%%%%%%% 
%Comandos especiais relacionados à classe 
%%%%%%%%%%%%%%%%%%%%%%%%%%%%%%%%%%%%%%%% 
\autor{Gabriel Kalil}
\titulo{Princípios da Igreja Multiplicadora aplicados na vida de profissionais em movimento global para abertura de frentes evangelizadores e/ou fortalecimento de igrejas} 
\data{2021} 
\local{CURITIBA} 
\preambulo{Trabalho de Conclusão de Curso, apresentado ao Curso de Bacharel em Teologia a distância pelas Faculdades Batista do Paraná, como requisito parcial para a obtenção do título de Bacharel em Teologia.\\\\Orientador: Prof. M.e Marcos Paulo Ferreira} 
\orientador{Marcos Paulo Ferreira} 
\tipotrabalho{monografia} 


%%%%%%%%%%%%%%%%%%%%%%%%%%%%%%%%%%%%%%%%%% 
% Pacotes de citações
\usepackage[brazilian,hyperpageref]{backref}	 % Paginas com as citações na bibl
\usepackage[alf]{abntex2cite}	% Citações padrão ABNT

%========  FABAPAR  =======
%\pagenumbering{arabic}

% Fonte Arial padrão para texto
\usepackage{uarial}

\setlength\afterchapskip{\lineskip}
\renewcommand{\baselinestretch}{1.5}
\setlength{\ABNTEXcitacaorecuo}{4cm}
\setlrmarginsandblock{2cm}{2cm}{*}
\setulmarginsandblock{3cm}{3cm}{*}
\checkandfixthelayout


%========  FABAPAR  =======

% CAPA CUSTOM
\renewcommand{\imprimircapa}{%
  \begin{capa}%
    \center
    \ABNTEXchapterfont\Large FACULDADES BATISTA DO PARANÁ

    \vspace*{5,5cm}
    \MakeUppercase{{\ABNTEXchapterfont\large\imprimirautor}}
    \vfill
    \begin{center}
	\MakeUppercase{\ABNTEXchapterfont\bfseries\large\imprimirtitulo}
    \end{center}
    \vfill
    \large\imprimirlocal

    \large\imprimirdata
    \vspace*{1cm}
  \end{capa}
}
% FOLHA DE ROSTO CUSTOM
\makeatletter
\renewcommand{\folhaderostocontent}{
	  \begin{center}
    \MakeUppercase{{\ABNTEXchapterfont\large\imprimirautor}}
    \vspace*{\fill}\vspace*{\fill}
    \begin{center}
    \MakeUppercase{\ABNTEXchapterfont\bfseries\large\imprimirtitulo}
    \end{center}
    \vspace*{\fill}
    \abntex@ifnotempty{\imprimirpreambulo}{%
     \hspace{.45\textwidth}
     \begin{minipage}{.5\textwidth}
       \SingleSpacing
        \imprimirpreambulo
      \end{minipage}%
      \vspace*{\fill}
    }%
  {\abntex@ifnotempty{\imprimirinstituicao}{\imprimirinstituicao
\vspace*{\fill}}}
    {\large\par}
    \abntex@ifnotempty{\imprimircoorientador}{%
      {\large\imprimircoorientadorRotulo~\imprimircoorientador}%
    }%
    \vspace*{\fill}
    {\large\imprimirlocal}
    \par
    {\large\imprimirdata}
    \vspace*{1cm}
  \end{center}
}
\makeatother
%%%%%%%%%%%%%%%%%%%%%%%%%%%%%%%%%%%%%%%%%% 
%Início do corpo do texto 
%%%%%%%%%%%%%%%%%%%%%%%%%%%%%%%%%%%%%%%%%%% 
\begin{document} 

\imprimircapa 
\imprimirfolhaderosto 

%%%%%%%%%%%%%%%%%%%%%%%%%%%%%%%%%%%%%%%%% 
% Início do resumo - obrigatório 
%%%%%%%%%%%%%%%%%%%%%%%%%%%%%%%%%%%%%%%%%%%%%%%%%%%%%%%%%%%%%% 
\begin{resumo} 
	Jesus iniciou um movimento global de discípulos a partir da Grande Comissão, apresentada em Mateus 28. Os primeiros discípulos começaram a fazer discípulos, mas o "ir" foi iniciado a partir de uma perseguição, que os dispersou pelo mundo. Em Antioquia da Síria, a primeira comunidade entre não judeus foi fundada por discípulos leigos a partir do seu testemunho e proclamação do Evangelho, mensagem que alcançou os nossos tempos. Hoje também, muitos perfis de pessoas estão em movimento global, dentre os quais, discípulos com perfil profissional. A Igreja Multiplicadora, movimento da Junta de Missões Nacionais da Convenção Batista Brasileira, busca o resgate de princípios bíblicos de discipulado e evangelização pode, através de seus princípios, inspirar esses discípulos profissionais em trânsito global a cumprirem o seu chamado, assim como os discípulos de Antioquia. O presente trabalho se propõe a apresentar esses três elementos: discípulos leigos fundando a igreja de Antioquia, os movimentos globais de migrantes com ênfase nos profissionais, e os princípios da Igreja Multiplicadora aplicados a discípulos com esse perfil profissional global.
  
\vspace{\onelineskip} 
\noindent 
\textbf{Palavras-chave}: Discipulado; Antioquia da Síria; Evangelização; Igreja Multiplicadora; Profissionais globais; migrantes.
\end{resumo} 

%%%%%%%%%%%%%%%%%%%%%%%%%%%%%%%%%%%%%%%%%%%%%%%%%%%%%%%%%%%%%%%%%
% Início do abstract - obrigatório 
%%%%%%%%%%%%%%%%%%%%%%%%%%%%%%%%%%%%%%%%%%%%%%%%%%%%%%%%%%%%%% 
\begin{resumo}[Abstract] 
\begin{otherlanguage*}{english}
	Jesus started a global movement of disciples from the Great Commission, presented in Matthew 28. The first disciples started to make disciples, but going to other nations initiated out of persecution, which dispersed them throughout the world. In Antioch of Syria, the first community among non-Jews was founded by lay disciples based on their testimony and proclamation of the Gospel, a message that has reached us. Today too, many people are on the move globally, among them, disciples with a professional profile. The Multiplier Church, a movement of the National Missions Organization of the Brazilian Baptist Convention, seeks to rescue biblical principles of discipleship and evangelization and can, through its principles, inspire these professional disciples in global transit to fulfill their calling, after the disciples of Antioch. The present work presents these three elements: lay disciples founding the church of Antioch, the global migrant movements with an emphasis on professionals, and the principles of the Multiplier Church applied to disciples with this global professional profile.

\vspace{\onelineskip} 

\noindent \textbf{Keywords}: Discipleship; Antioch of Siria; Evangelism; Multiplying Church movement; global professionals; migrants.
\end{otherlanguage*} 
\end{resumo} 
%%%%%%%%%%%%%%%%%%%%%%%%%%%%%%%%%%%%%%%%%%%%%%%%%%%%%%%%%%%%%%%%% 
% Fim do abstract - obrigatório 
%%%%%%%%%%%%%%%%%%%%%%%%%%%%%%%%%%%%%%%%%%%%%%%%%%%%%%


\tableofcontents* % imprime sumário


\chapter*{Introdução}
\addcontentsline{toc}{chapter}{INTRODUÇÃO}  

O movimento iniciado por Jesus, relatado nos Atos dos Apóstolos tem como primeira expressão a pregação pública de Pedro, onde muitos irmãos são acrescentados, e surge uma igreja que crescia diariamente. Em uma primeira fase, a pregação de massa é o instrumento de propagação da mensagem do Evangelho. Com o tempo, outros instrumentos são acrescentados, e a igreja já organizada, inicia o preparo, envio e sustento de missionários dedicados primordialmente à pregação do Evangelho e plantação de novas Igrejas. Entre estes dois momentos, existe a dispersão dos cristãos motivada pela perseguição relatada em Atos 11, movimento a partir do qual também surgiram igrejas com a de Antioquia (Atos 11:19-30). Tal igreja, iniciada por cristãos comuns, é considerada a primeira igreja com foco nos gentios e base de lançamento para o projeto missionário do apóstolo Paulo.
	
Os dias atuais também veem um movimento de dispersão de cristãos leigos pelo mundo. Desta vez, no entanto, existem vários outros fatores envolvidos:  funcionários expatriados de empresas globais, estudantes em intercâmbio, militares em serviço, refugiados, famílias e indivíduos em busca de melhores condições de vida, entre outros. 

O presente trabalho tem como tema "Princípios da Igreja Multiplicadora aplicados na vida de profissionais em movimento global para abertura de frentes evangelizadores e/ou fortalecimento de igrejas". O objetivo é buscar uma correlação entre os primeiros discípulos leigos (ou seja, não formalmente preparados e enviados) apresentados na história da Igreja, os que chegaram em Antioquia da Síria. Além disso, o trabalho quer mostrar sua relação com discípulos de nossos tempos também em movimento global. A partir dessa relação, procura-se apresentar o movimento da Igreja Multiplicadora (iniciado a partir da Junta de Missões Nacionais da Convenção Batista Brasileira), e como os princípios desse movimento podem ser utilizados por esses discípulos em dispersão moderna, para, assim com os discípulos de Antioquia, cumprir seu chamado recebido de Jesus em Mateus 28. A metodologia da revisão bibligráfica define a maneira como o assunto é explorado.

Na primeira parte, o trabalho faz um breve desenvolvimento a respeito do significado da Igreja em seus sentidos universal e local, utilizando como referência vários autores na área de teologia sistemática como BAVINCK, BERKHOF e SEVERA. O trabalho então, coloca a igreja em seu sentido local em evidência, citando seus fundadores e enfatizando dentre suas características, que eram discípulos, profissionais, leigos, e em movimento, com base no trabalho de STETZER. A partir daí, o trabalho apresenta o conceito de discipulado que era conhecido na época de Jesus, e mostrando como o conceito ganhou contornos teológicos em pouco tempo, distinguindo os discípulos de Jesus e vinculando esse discipulado à identificação com o Senhor e agregação a uma comunidade local. Primariamente NEWTON, SHIRLEY dão o necessário embasamento teórico, seguidos por WILKINS, BRANDAO e WAN. A seção sobre "Evangelização Discipuladora" termina a construção do conceito de discipulado que o trabalho quer utilizar, convidando STOTT para apresentar o aspecto evangelístico do mesmo.

O movimento de evangelismo discipulador alcançou a cidade de Antioquia da Síria, onde os discípulos começaram a ser vistos de maneira distinta dos judeus, e portanto chamados cristãos. Essa comunidade nascida entre os gentios era composta de pessoas de muitas origens, promovido pelo fato de Antioquia ser um ponto de convergência Ocidente/Oriente. Fundada por discípulos leigos, a comunidade dos cristãos trazia para a mesma mesa um aristrocrata, um ex-fariseu, um levita, um judeu helênico e um africano. GREEN e STOTT nos ajudam a observar as características dessa comunidade multi cultural de leigos, que lançou as bases para a evangelização dos gentios. 

Fazendo um paralelo com o perfil dos fundadores da igreja de Antioquia, o trabalho estuda os movimentos globais de pessoas da atualidade, dando especial enfoque a um subgrupo, os profissionais qualificados. Usando como referencial dados de 2020 do relatório sobre migração global da Organização Internacional de Migração, o trabalho cita os principais movimentos migratórios modernos, bem como os países que mais enviam e recebem migrantes nos dias de hoje. O trabalho prossegue apresentando uma categoria migratória específica, a diáspora moderna, e, a partir dos tralhos de BRAZIEL, REIS E WAN, esclarece a importância e escala desse tipo de migração. Para além disso, CHISWICK e BAUER nos mostram o perfil mais requisitado de migrantes em diáspora, os profissionais qualificados. 

Na última etapa, o trabalho apresenta a Igreja Multiplicadora e seus princípios, fazendo diveras ligações entre esses princípios e os discípulos profissionais em movimento global. O trabalho conclui apresentando maneiras pelas quais esses profissionais podem ser inspirados no processo de evangelização, discipulado e suporte a igrejas locais.

\chapter{O surgimento da Igreja e os discípulos em movimento evangelizador}

O termo "igreja" tem sua origem no grego (ekklesia), uma palavra comum na época de Jesus. A mesma palavrava era utilizada no mundo grego como uma assembleia de pessoas, reunidas por motivos relacionados à cidade, sendo eles de origem religiosa, cível ou social, ou seja, uma palavra de uso comum\cite[p. 317]{zac}. Entre os judeus oriundos da Diáspora, "ekklesia" que inicialmente se utilizava como sinônimo da palavra "synagoge", logo tomou um sentindo distinto. Enquanto "synagoge" continuou a ser usada para identificar o encontro realizado nas sinagogas, "ekklesia" se tornou a palavra que identificava a comunidade daqueles que Deus havia chamado para a salvação\cite[p. 485]{bavinck}. 

No Novo Testamento, a palavra "ekklesia" é utilizada em 95\% dos casos diretamente relacionada aos cristãos, ou seja, ao grupo de seguidores de Jesus Cristo. O sentido é de congregação ou assembleia dos seguidores de Cristo, ou, nos termos dos reformadores, "a comunidade dos que creem em Cristo e são santificados nele" \cite[p. 318]{zac}. A igreja é referenciada no Novo Testamento tanto no sentido universal como local\cite[p. 318]{zac}.

\section{Igreja em sentido universal}

Jesus foi a primeira pessoa a utilizar a palavra igreja no Novo Testamento, aplicando-a ao grupo que o cercava (Mateus 16:18), ou seja, a igreja do Messias, o verdadeiro Israel de Deus \cite[p. 911]{berkhof}. A partir daí, o conceito de igreja é desenvolvido para significar o conjunto de todos os crentes salvos por Jesus Cristo em todos os tempos e lugares, uma entidade de aspecto espiritual \cite[p. 318]{zac}. Jesus é o cabeça deste grupo:

\begin{citacao}“Deus colocou todas as coisas debaixo de seus pés e o designou cabeça de todas as coisas para a igreja, que é o seu corpo, a plenitude daquele que enche todas as coisas, em toda e qualquer circunstância.” (Efésios 1:22,23).
\end{citacao}

A igreja universal abrange o conjunto das pessoas humanas que um dia estarão diante de Deus como a noiva imaculada de Cristo, uma grande assembleia de testemunhas desde a criação do mundo \cite[p. 607]{bavinck}. A essa igreja Cristo amou e a si mesmo se entregou por ela (Efésios 5:25), "aqueles que quando do momento de sua regeneração (Tito 3:3-6), individualmente colocam a sua fé no Senhor Jesus como seu Salvador (Atos 16:31)" \cite[p. 319]{zac}.

\section{Igreja em sentido local}

Além do seu sentido universal como entidade espiritual, a igreja também se manifesta como grupo local de pessoas que se encontram regularmente para edificação, conforme Hebreus 10:25 e Efésios 4:12. Esses grupos locais são formados por pessoas que confessam sua fé em Cristo Jesus e que compartilham de um conjunto de doutrinas cristãs, tendo um ponto de referência geográfico comum para seus encontros. A maioria das epístolas do Novo Testamento e as sete cartas do Apocalipse foram enviadas para igrejas locais, sejam elas grupos únicos de uma cidade ou pequenos grupos reunidos nas casas \cite[p. 320]{zac}. 

Um exemplo famoso é o pequeno grupo que se torna igreja, iniciado na casa de Aquila e Priscila, cujos nomes são mencionados diversas vezes no Novo Testamento. Eles eram pessoas comuns, possivelmente donos do seu próprio negócio. Prováveis fundadores da igreja em Éfeso, são citados em várias cidades e iniciaram igrejas em suas residências tanto em Roma como em Éfeso. Apesar de não existirem registros sobre a fundação da igreja de Roma, pode-se assumir que o casal ajudou a inicia-la \cite[p. 54]{stetzer}.

\section{A igreja e os movimento de evangelização discipuladora}

Finalizando seu ministério como "filho do homem", Jesus outorgou poder a seus discípulos e os comissionou para replicar seu modelo e continuar o que ele começou. O evangelista Mateus registra esse evento conhecido como a Grande Comissão: 

\begin{citacao}
"Jesus se aproximou deles e disse: "Toda a autoridade no céu e na terra me foi dada. Portanto, vão e façam discípulos de todas as nações, batizando-os em nome do Pai, do Filho e do Espírito Santo. Ensinem esses novos discípulos a obedecerem a todas as ordens que eu lhes dei. E lembrem-se disto: estou sempre com vocês, até o fim dos tempos" (Mateus 28:18-20).
\end{citacao}

O comissionamento de Jesus se extende nas dimensões do espaço (até os confis da terra) e tempo (até o fim dos tempos). Dentro dessa dinâmica, a ordem é clara: fazer discípulos.  

\subsection{Aprendizes} 

No tempo de Jesus, discípulo (do grego mathetes) era uma palavra conhecida, e literalmente significava aprendiz. A palavra matemática, por exemplo, vem da mesma raiz de mathetes e significa "raciocínio seguido por um comportamento", ou seja, um conhecimento aplicado \cite[p. 15]{gtsm}. Assim também os discípulos na época de Jesus eram mais do que estudantes absorvendo conhecimento intelectual. Ao contrário, aprendiam e desenvolviam o trabalho de seu mestre \cite[p. 15]{gtsm}. No contexto do Novo Testamento, o mesmo termo é utilizado para designar os discípulos de João Batista (Mateus 9:14), dos fariseus (Marcos 2:18), além dos discípulo do próprio Jesus \cite[p. 59]{brandao}. No mundo greco romano, discípulos permeavam a sociedade, desde discípulos de filósofos como Sócrates até discípulos de rabinos proeminentes como Shammai e Hilel. Tanto uns como outros discípulos, eram encharcados pelos conhecimentos de seus mestres enquanto os seguiam \cite[p. 209]{shirley}.

A palavra discípulo, no entanto, foi tendo seu significado atualizado com o passar dos anos. Entre a primeira menção da palavra nos evangelhos (Mateus 5:1) e sua última menção em Atos 21:16, muito mudou em seu sentido. Discípulo no primeiro século era um seguidor que aprendia, gradualmente realizava enquanto aprendia e finalmente ensinava nos mesmos moldes. A conexão entre os aspectos cognitivo e prático já existia \cite[p. 105]{wilkins}. Começando com a narrativa de Atos dos Apóstolos e continuando pelo Novo Testamento, a palavra discípulo ganhou uma conotação teológica e específica, designando aqueles que se convertiam ao Evangelho, que compunham a "multidão dos discípulos" (Atos 6:2). Ainda em Atos dos Apóstolos, discípulos eram aqueles que criam em Jesus, eram salvos e se integravam ao Corpo de Cristo (At 9.26)\cite[p. 59-60]{brandao}. Em Antioquia os discípulos foram chamados de cristãos pela primeira vez. Essa designação, provavelmente atribuída por pessoas de fora da comunidade, tinha um aspecto inicialmente pejorativo, identificando uma nova "seita" entre tantas. No entanto, indicava que uma nova fé emergia distintamente do judaísmo, uma nova comunidade de maioria de crentes gentios e com uma nova composição étnica \cite[p. 90]{wan_diaspora_2011}. A palavra discípulo nesse contexto, portanto, ao mesmo tempo em que inclui o aspecto teológico do novo nascimento e identificação com a pessoa de Jesus e incorporação em seu Corpo, retém também o sentido de aprendizado vivencial, teoria e prática. Discípulos são os salvos por Jesus Cristo que passam a segui-lo imitando a seu Senhor.

\subsection{Discipulando como o Mestre}

%TODO Chamada ao estilo de vida ao fazer discípulos e estar em movimento. Pensar em fundamento exegético?

O novo fazer discípulos, comandado por Jesus em Mateus 28 inclui diversos aspectos no processo de imitar o modelo do Mestre. As dimensões de evangelismo (ir), agregação ao Corpo de Cristo dos convertidos (batizar), e o posterior aperfeiçoamento desses discípulos (ensinar) são partículas indissociáveis. Além disso, o mandato para fazer outros discípulos ainda é o mesmo para os discípulos de Jesus de nossos tempos e os que ainda virão. Fazer discípulos não é uma etapa do Discipulado e uma ação subsequente ao "agregar" ou "batizar", mas sim resultado direto do chamado de Jesus ao discipulado \cite[p. 64]{brandao}, conforme narrado no evangelho de Marcos: "Jesus lhes disse: "Venham! Sigam-me, e eu farei de vocês pescadores de gente". No mesmo instante, deixaram suas redes e o seguiram."(Marcos 1.17,18).

\subsection{Evangelização Discipuladora}

O cumprimento da promessa no dia de Pentecostes e a subsequente formação da primeira comunidade dos que pertenciam ao "Caminho" se confirmava, pois "a cada dia, o Senhor lhes acrescentava aqueles que iam sendo salvos" (Atos 2:47). John Stott reconhecendo a característica evangelística dessa igreja, afirma que o crescimento diário não foi um evento esporádico. A igreja embora ainda não organizasse programas de missões, na mesma medida em que adorava ao Senhor diariamente (Atos 2:46), também exercitava o seu testemunho. Tanto a adoração quanto a proclamação fluem naturalmente de corações cheios do Espírito Santo \cite[p. 118,119]{stott}. 

O Espírito Santo é missionário e deu início a uma igreja evangelizadora nos moldes de Jesus. O testemunho diário dessa comunidade foi um veículo para que a igreja crescesse \cite[p. 78]{stott} sob a fundação apostólica, ou seja, os primeiros discípulos. Observamos essa característica logo após o evento de Pentecostes, quando Pedro e João foram instrumentos da  cura milagrosa de um aleijado. Após serem ameaçados pelo conselho de líderes, foram libertos. Ao retornarem para o lugar de onde haviam vindo, os outros irmãos, identificados com a recente experiência pela qual passaram Pedro e João, juntos levantaram a voz em clamor a Deus. Em resposta a esse clamor e capacitados pelo Espírito Santo, continuaram a pregar as boas novas corajosamente (Atos 4:21-31). Os primeiros discípulos aprenderam com o mestre durante três anos, e agora manifestavam aspectos essenciais do estilo de Jesus: relacionamento (estavam juntos), compaixão (Pedro e João acabavam de voltar da prisão), oração (clamavam juntos), evangelização (pregavam corajosamente). Os primeiros seguidores de Jesus iniciaram então uma comunidade com os mesmos elementos que haviam aprendido de seu mestre: "todos se dedicavam de coração ao ensino dos apóstolos, à comunhão, ao partir do pão e à oração" (Atos 2:42). A evangelização e ensino através de relacionamentos construiu uma comunidade integrada. Enquanto se relacionavam, os discípulos faziam outros discípulos que aprendiam o estilo de evangelismo de Jesus: o evangelismo discipulador.

%TODO Aqui poderia expandir e trabalhar melhor estes princípios (pós cap 2). (4 princípios Atos 2:42) => Formação de líderes também.

% O conceito de movimento e espontaneidade é que tem que aparecer aqui. Essa visão é que tem que estar aqui.
% Tem que virar um artigo que igrejas e pessoas podem usar. Se Deus me mandar para Pequim, é uma missão estar lá.
% Christopher Wright (a missão de Deus), diz que Deus é um Deus em movimento (ver pespectivas), como Jesus ele põe o povo em movimento.

\subsection{Espontâneo e em movimento}

A partir do evento da vinda do Espírito Santo narrado em Atos 2, se seguiu uma série de eventos que nos apontam para um novo movimento entre discípulos: no desenrolar da normalidade dessa nova vida, tiveram muitas oportunidades de distribuírem a graça de Deus (1 Pedro 4:10), testemunhando do que viram e ouviram (1 João 1:3), apresentando a Jesus e o Reino de Deus. Uma atividade comum, por exemplo, era o ir ao templo para orações, prática normal da vida de um judeu piedoso da época. Pedro e João, em um desses momentos, narrado em Atos 3:1 se deparam com uma cena também comum: um aleijado que todos os dias era colocado ao lado da porta do templo para pedir esmolas, evento que certamente testemunharam muitas vezes. Nessa situação de rotina, o impulso do Espírito Santo através de dois discípulos sensíveis, operou uma cura milagrosa e atraiu uma multidão, proporcionando a Pedro uma plataforma evangelística, que em poucas palavras já apresentava Jesus como o messias esperado, sua natureza divina e ressurreição:

\begin{citacao}
	"Pedro, percebendo o que ocorria, dirigiu-se à multidão. “Povo de Israel, por que ficam surpresos com isso?”, disse ele. “Por que olham para nós como se tivéssemos feito este homem andar por nosso próprio poder ou devoção? Pois foi o Deus de Abraão, de Isaque e de Jacó, o Deus de nossos antepassados, quem glorificou seu Servo Jesus, a quem vocês traíram e rejeitaram diante de Pilatos,	apesar de ele ter decidido soltá-lo. Vocês rejeitaram o Santo e Justo e, em seu lugar, exigiram que um assassino fosse liberto. Mataram o autor da vida, mas Deus o ressuscitou dos mortos. E nós somos testemunhas desse fato!" (Atos 3:12-15).
\end{citacao}

O próprio Jesus, na normalidade da vida de seus discípulos, proporcionou oportunidades para que as boas novas fossem proclamadas.

Em paralelo a esse movimento, outro já havia sido iniciado no evento da descida do Espírito Santo: pessoas de várias nações testemunharam juntamente com os judeus o evento que marcou a história, e deu para Pedro outra oportunidade para evangelização em massa. Dessa vez, um grupo estupefato de 
\begin{citacao}"partos, medos, elamitas, habitantes da Mesopotâmia, da Judeia, da Capadócia, do Ponto, da província da Ásia, da Frígia, da Panfília, do Egito e de regiões da Líbia próximas a Cirene, visitantes de Roma (tanto judeus como convertidos ao judaísmo), cretenses e árabes" (Atos 2:9-11).\end{citacao} receberam e puderam compartilhar as boas novas de salvação na volta para seus locais de origem. A normalidade da vida e a falta de planos não impediram o Espírito Santo de executar a Sua obra. O mesmo Jesus que curava, ensinava e discipulava na normalidade da vida agora também operava na normalidade da vida de seus discípulos.

\section{A primeira comunidade cristã entre os gentios}

O movimento de evangelismo discipulador iniciado por Jesus, levou os discípulos a avançarem no estabelecimento de novas comunidades. Do ponto de vista cultural, a grande comissão começou a partir da evangelização dos judeus, mas a ordem de Jesus incluía também os gentios (por todo o mudo). O alcance dos gentios aconteceu de várias maneiras. Em especial, quando alguns discípulos que, fugindo da perseguição capitaneada pelos líderes judeus, chegaram a Antioquia da Síria, onde começaram a pregar as boas novas a pessoas de fala grega. A comunidade de Antioquia teve um papel fundamental no cumprimento da grande comissão. É a partir de lá que foi lançado o projeto missionário de Paulo entre os gentios, conforme (Atos 11:19-26,13:1-3). Nesse relato, o evangelista Lucas menciona pela primeira vez a pregação do Evangelho para pessoas totalmente pagãs, uma mudança decisiva na história da igreja. A inovação foi aceita pela igreja de Jerusalém e confirmada pelo envio de Barnabé a Antioquia. Este, fora comissionado para comunicar à nova comunidade a decisão de Jerusalém, de que os gentios convertidos não precisavam cumprir os ritos da Lei de Israel, inclusive a circuncisão. A fé e o batismo eram suficientes para identificá-los com Jesus, independentemente de sua origem étnica. \cite[p. 198]{green}.

Antioquia era a capital da província da Síria, uma cidade extremamente cosmopolita. Apesar de ser de fundação grega, seus estimados 500.000 habitantes na época, tinham origens variadas. A uma grande colônia de Judeus, se somavam habitantes oriundos da Pérsia, Índia, China, latinos e outros. Gregos, judeus, orientais e romanos formavam uma combinação que levou o historiador Josephus a considerá-la a terceira cidade do império, atrás somente de Roma e Alexandria \cite[p. 185]{stott}. A cidade apresentava um ambiente interessante para o alcance dos gentios, uma vez que as barreiras entre os judeus e eles eram muito tênues. Haviam muitos convertidos ao judaísmo, e os judeus, gozavam direitos de cidadãos. Por ser um dos grandes centros comerciais e uma das maiores cidades do império, a cidade desenvolvera relações comerciais por todo o mundo, o que facilitava o trânsito de pessoas num contexto multi nacional e multi cultural, literalmente um local de convergência entre o Ocidente e o Oriente. Vários sistemas de crença também se encontravam ali, desde o culto a Zeus e outros deuses gregos, Baal (de origem síria), religiões de mistério, entre outros.\cite[p. 166,167]{green}. Com um contexto cultural e religioso tão variado, Antioquia da Síria poderia ser facilmente confundida com capitais Europeias como Londres ou Berlim da atualidade.

%\subsection {Relacional desde o princípio}

A comunidade de discípulos que nasceu em Antioquia da Síria, tinha um estilo de vida que imitava o discipulado de Jesus. O foco nos relacionamentos estava presente de uma maneira muito intensa. A nova comunidade conseguia trazer para o mesmo local de comunhão, pessoas de origens absolutamente distintas. Povos, culturas, visões políticas e pressupostos religiosos estavam juntos à mesa e compartilhavam do mesmo pão. A igreja de Antioquia da Síria

\begin{citacao}
era uma igreja tão comprometida com a comunhão que judeus e gentios convertidos à fé quebraram barreiras seculares e comiam à mesma mesa. Era uma igreja em que pessoas como Manaém (um aristocrata), Paulo (um ex-fariseu super-rígido), Barnabé (um levita de Chipre e antigo proprietário de um campo), Lúcio (um judeu helênico de Cirene) e “Simeão, chamado Níger” (muito provavelmente um africano) podiam trabalhar juntos em uma liderança harmoniosa de crentes. Essa comunhão amorosa não estava limitada a Antioquia. Paulo agradece a Deus pelo amor dos tessalonicenses (1Ts 1.3.) e ora para que seu amor possa transbordar sempre mais em favor de todas as pessoas (1Ts 3.12). Foi o próprio Deus que implantou essa coesão interna, e nesse sentido Paulo nem precisava mencioná-la (1 Tessalonicenses 4:9) \cite[p. 261]{green}.
\end{citacao}

A missão dada por Jesus começava a se desenrolar entre outros povos. Os discípulos em movimento haviam encontrado um solo fértil fora do seu local de origem, longe de suas raízes culturais, fora do conforto e proteção do ambiente que lhes era familiar. No próximo capítulo observaremos alguns movimentos de dispersão modernos, que embora nem sempre causados pelos mesmos motivos, tem um potencial similar de propagação do Evangelho.

\chapter{Profissionais em Trânsito}

Movimentos como o que deu origem à igreja de Antioquia continuam a acontecer nos dias de hoje. Nos dias de hoje, esses movimentos são conhecidos como movimentos migratórios. Existem várias definições existem sobre o termo com variações relacionadas a aspectos geográficos, legais, políticos, metodológicos, temporais, entre outros\cite{iom2020}. Para fins de simplificação, escolhemos a definição utilizada pelo Departamento de Estatísticas de Migração Internacional das Nações Unidas, que classifica como migrantes quaisquer pessoas que mudaram seu país de residência. Mais especificamente, migrantes são não refugiados vivendo fora de seu país de origem. 

\section {Movimentos Migratórios Modernos}

Dados históricos mostram que os movimentos migratórios, estão em grande parte relacionados em um nível global, com transformações nas áreas econômica, social, política e tecnológica. Conforme os processos de globalização se desenvolvem, essas transformações geram impacto nas famílias, ambientes de trabalho e sociedade. Os movimentos migratórios nem sempre são resultado de processos pacíficos. Nos últimos dois anos, especialmente, muitos desses eventos tiveram origens traumáticas. Conflitos (como o da Síria, Iêmen, República Democrática do Congo, Sudão do Sul), violência extrema (como o do Povo Rohingya que foi forçado a procurar abrigo em Bangladesh) ou instabilidades políticas ou econômicas (como na Venezuela) marcaram os últimos anos. As mudanças do meio ambiente também tiveram um impacto na mobilidade humana. Migrações em larga escala desencadeadas por catástrofes climáticas e ambientais ocorreram em diversas partes do mundo em 2018 e 2019, em países como Filipinas, China, Índia, e Estados Unidos da América \cite[p. 19]{iom2020}.

De acordo com a Organização Internacional de Migração, haviam em 2019, 272 milhões de migrantes globalmente, o equivalente a 3.5\% da população mundial, 74\% deles em idade produtiva (de 20-64 anos de idade). O número de migrantes internacionais estimado vem crescendo ao longo das últimas cinco décadas, passando de 2.3\% da população mundial (84.4 milhões de pessoas) em 1970 para 3.5\% em 2019, ou seja, uma em cada 30 pessoas. De acordo com o mesmo relatório, o país que mais recebeu migrantes foram os Estados Unidos da América, posição essa que vem sendo mantida desde 1970. Desde lá, o número de migrantes quadruplicou, de menos de 12 milhões de migrantes em 1970 para quase 51 milhões em 2019. A Alemanha segue os EUA na lista dos países que mais recebem migrantes, com aproximadamente 13.1 milhões em 2019 \cite[p. 21]{iom2020}. Um aspecto importante da distribuição internacional de migrantes, é que aproximadamente dois terços deles, residiam em países de alta renda em 2019, ou seja, países cujo Produto Interno Bruto per Capita é igual ou maior do que 12.536 dólares americanos \cite{world_bank_country_lending_groups}. Observou-se também, que embora migrantes internacionais tenham uma tendência de se deslocar para países de alta renda, suas origens são diversas.  Dentre os países que mais enviam migrantes para outros países, se destacam a Índia com 17.5 milhões de migrantes enviados, seguida pelo México (11.8 milhões) e China (10.7 milhões). 

\section{Diáspora}

Dentre os movimentos migratórios conhecidos na história, uma categoria é de especial interesse do ponto de vista da propagação do conhecimento de Deus entre os povos: a diáspora. O termo diáspora tem sido cada vez mais usado por antropólogos, teóricos literários e críticos culturais, para descrever as migrações em massa e grandes deslocamentos de pessoas da segunda metade do século vinte. Em especial, o termo descreve movimentos de independência de áreas previamente colonizadas, refugiados deixando áreas de conflito e fluxos de migração econômica pós Segunda Guerra Mundial \cite[p. 11]{braziel}. No contexto moderno, o capital se tornou global, e portanto, termos como "globalização" e "capital global" se tornaram relevantes no estudo e definições sobre diáspora. A partir do estabelecimento de acordos de comércio internacional como o NAFTA e GATT, e a formação de alianças transnacionais como a União Europeia, adentramos um período histórico onde empresas multinacionais exportam indústrias e postos de trabalho, e tem a possibilidade de movimentar estes postos de trabalho de um país para o outro. As fronteiras entre local e global, entre cidadãos nativos e diáspora, são cada vez menos distintas \cite[p. 19]{braziel}. Para além disso, cabe expandir o conceito clássico de diáspora (diáspora do povo Judeu), de modo a incluir as três principais ondas de diáspora que influenciaram o mundo: 

\begin{itemize}
	\item O período clássico dos judeus antigos e a diáspora grega;
	\item O período moderno de escravidão e colonização, dividido em (1) expansão do capital europeu entre 1500-1814, (2) a Revolução Industrial de 1815-1914 e (3) o período entre a Primeira e a Segunda Guerras mundiais;
	\item O período contemporâneo, logo após a Segunda Guerra mundial até os dias de hoje.
	\cite[p. 41-60]{reis}
\end{itemize}

Os seres humanos sempre se deslocaram espacialmente devido a fatores como guerra, fome, perseguição política ou religiosa. A número de pessoas se deslocando em larga escala, no entanto, se intensificou desde o último século, o que resultou em um grande número de diásporas. Várias diásporas de nível global aconteceram no século XX, como por exemplo a diáspora dos judeus na Europa causada pelo nazismo, a diáspora palestina causada pelo reestabelecimento do estado de Israel em 1948, conflitos políticos na Nicarágua, El Salvador, Guatemala, Honduras, a revolução iraniana de 1979, entre outros \cite[p. 25]{wan_diaspora_2011}.

O século XX foi palco de muitos movimentos de diáspora com crescente frequência e complexidade. Estima-se que a população de diáspora no ano de 2000 era de 175 milhões de pessoas, deslocando-se geograficamente do norte para o sul, e do leste para o oeste em direção aos países mais ricos do mundo, que embora abriguem 16\% da população mundial, receberam mais de 33\% da população de migrantes \cite[p. 26]{wan_diaspora_2011}.

\section{Diáspora profissional}

A diáspora que acontece nos dias atuais, é em grande parte resultado da demanda por profissionais qualificados em países desenvolvidos. Barry Chiswik, escrevendo para o Instituto de Economia do Trabalho em Bohn na Alemanha (IZA), cita a crescente demanda por profissionais qualificados nos países da OECD (Organização para Cooperação Econômica e Desenvolvimento), da qual fazem parte uma grande parte dos países desenvolvidos da Europa, Américas, Ásia e Oceania. A globalização da economia mundial é um produto de vários elementos, sendo que 

\begin{citacao} um elemento essencial tem sido o desenvolvimento econômico de vários países menos desenvolvidos no passado. Como resultado, muitos produtos previamente produzidos por trabalhadores de baixa qualificação nos países da OECD, são agora produzidos em países menos desenvolvidos. A globalização da economia mundial alterou a especialização da produção internacional. Isso elevou a demanda por profissionais de alta qualificação nas economias avançadas, mas reduziu a demanda por profissionais de baixa qualificação nas economias cujos postos de trabalhos foram transferidos para outros países \cite[p. 3]{chiswick_high_2005}. \end{citacao}

Por consequência, países como Canada, Austrália, Estados Unidos e Japão tem desenvolvidos políticas migratórias com base em um sistema de pontos que privilegia profissionais qualificados \cite[p. 6]{chiswick_high_2005}. A pesquisa recente realizada pelo IZA (Instituto de Economia do Trabalho na Alemanha) sobre a demanda por profissionais altamente qualificados envolvendo parte da Alemanha, França, Reino Unido e Holanda, mostrou que as áreas de Tecnologia de Informação, Recursos Humanos e Finanças são os setores com maior percentual de migrantes de alta qualificação.\cite[p. 7]{bauer_demand_2004}. Outros setores da indústria também contribuem significativamente com a demanda por migrantes qualificados, como a indústria química e setores de manufatura \cite[p. 18]{bauer_demand_2004}.

Essa tendência parece ser confirmada pelos dados divulgados pelo "Escritório Federal para Migração e Refugiados da Alemanha", o BAMF\footnote{Bundesamt für Migration und Flüchlinge}, que publicou estatísticas recentes a respeito de um dos tipos de visto de trabalho na Europa, o "EU Blue Card".\footnote{O EU Blue Card é um visto específico para profissionais qualificados, disponível em todos os países da União Europeia.} A demanda por esse tipo de visto tem crescido consideravalmente. O número de Blue Cards na Alemanha quase triplicou em 5 anos, entre 2013 e 2018. Em 2018, pouco mais de 50\% dos vistos de trabalho desse tipo foram concedidos a migrantes da Índia, China, Rússia, Turquia e Brasil. Em comparação com o resto da União Europeia, a Alemanha foi o país distribuiu o maior número desse vistos para profissionais qualificados, cerca de 84,5\% do total do bloco\cite{bamf}.

A demanda global por profissionais qualificado na atualidade, pode ser também percebida da perspectiva da Grande Comissão. O comissionamento dado por Jesus pode ser realizado de diversas maneiras. Uma delas, a qual buscamos destacar, é através do desenvolvimento de competências profissionais que permitam ao discípulo de Jesus navegar nas águas das diásporas profissionais modernas, fazendo discípulos de todas as nações, até os confins da terra.

\chapter{Princípios da Igreja Multiplicadora}

O comissionamento de Jesus para uma vida de movimento e fazer novos discípulos dele, Jesus, continua sendo o mesmo que Pedro, Paulo, João e outros discípulos implementaram nas primeiras décadas da era moderna. A obediência a esse chamado exigiu muita criatividade para entregar uma mensagem de esperança para fora do contexto judeu. Em Antioquia da Síria, por exemplo, a primeira igreja entre os gentios foi iniciada a partir de uma série de inovações por parte dos discípulos que ali chegaram. Uma dessas inovações foi o crescente uso do termo "Senhor" para se referir a Jesus, como uma alternativa ao termo "Cristo", que carregava uma forte conotação ao Messias judeu. Jesus como "Senhor", o posicionava em contraste a outros senhores (deuses) do mundo helênico. O Jesus Senhor reivindicava para si exclusividade, era o único "Senhor" \cite[p. 170]{green}. Iniciativas como essa, evidenciam o zelo dos discípulos em obedecer ao mestre, implementando os princípios que aprenderam de maneira a alcançar a todos. Da mesma forma, discípulos profissionais em trânsito global, precisam adaptar os mesmos princípios aprendidos de Jesus e encontrar novas maneiras de contextualizar o fazer discipulos.

\section{Igreja Multiplicadora}

A Igreja Multiplicadora é uma visão nascida na Junta de Missões Nacionais da Convenção Batista Brasileira, e que busca desenvolver multiplicação intencional de discípulos baseada em princípios bíblicos como oração, evangelização discipuladora, plantação de igrejas, formação de líderes e compaixão e graça. Em outras palavras, 

\begin{citacao}sermos bênção a partir de onde estamos para alcançar as nações por meio de um grande movimento de multiplicação de discípulos e igrejas é o grande objetivo da visão de Igreja Multiplicadora \cite[p. 19]{freitas}.\end{citacao}

A visão e movimento da Igreja Multiplicadora, nasceu a partir da constatação de que é necessário um retorno ao estilo de vida da igreja primitiva, cuja história é registrada no livro de Atos dos Apóstolos nos primeiros dois capítulos. A chegada do Espírito Santo marcou o rumo das igrejas neotestamentárias, cujos princípios são preciso resgatar. O fazer discípulos precisa ser o estilo de vida das igrejas, e mais importante, cada discípulo de Jesus Cristo: 

\begin{citacao}O povo de Deus é um povo de bênção para as nações. Na verdade, afirma Paulo, essa é a Boa-Nova, o evangelho (Gl 3.8). Abençoar as nações é a missão declarada de Deus. É por essa razão que ele chama seu povo à existência, para ser o veículo da missão do Senhor no mundo histórico das nações \cite[p. 98]{wright_missao_2012}.\end{citacao}

Abençoar nações através de movimentos de discipulado não é uma ideia nova. Dawson Trotman, por exemplo, fundador do conhecido ministério Navegadores na década de 1930, tinha como objetivo espalhar as Boas-Novas por meio de relacionamentos intencionais, capacitando discípulos para a multiplicação. Tendo como base o cuidado um a um, inspirado em duplas como Paulo e Timóteo, Barnabé e Paulo, os Navegadores enfatizam a importância de investir tempo com novos crentes, individual e coletivamente (em pequenos grupos). Através desses relacionamentos, vida é gerada a partir de vida, através da oração, encorajamento, aconselhamento e ensino mútuo\cite[p. 21]{freitas}. O apóstolo Paulo instruiu seu discípulo Timóteo nos mesmos termos: "Você me ouviu ensinar verdades confirmadas por muitas testemunhas confiáveis. Agora, ensine-as a pessoas de confiança que possam transmiti-las a outros." (2a Timóteo 2.2).

Os princípios bíblicos resgatados pela Igreja Multiplicadora podem ser aplicados hoje, como o foram na igreja primitiva. Oração, Evangelização Discipuladora, Plantação de Igrejas, Formação de Líderes, Compaixão e Graça são tão relevantes como nos tempos de Paulo e Barnabé, e com potencial para inspirar e instrumentalizar discípulos profissionais em trânsito global a realizarem seu chamado. Em seguida estudaremos esses princípios.


\section{Oração}

Orar foi o primeiro movimento dos discípulos após o recebimento da Grande Comissão\cite[p. 28]{brandao}. A importância da oração como estilo de vida dos discípulos é vista na igreja de Atos dos Apóstolos. Ao retornarem para Jerusalém, os discípulos aguardaram a promessa do Espírito Santo e perseveravam em oração: "Todos eles se reuniam em oração com um só propósito, acompanhados de algumas mulheres e também de Maria, mãe de Jesus, e os irmãos dele." (Atos 1:14). O resultado foi uma proclamação intensa do evangelho, onde muitos foram adicionados à jovem comunidade e "todos se dedicavam de coração ao ensino dos apóstolos, à comunhão, ao partir do pão e à oração." (Atos 2:42). A Igreja Multiplicadora 

\begin{citacao}apresenta o princípio Oração, com dois objetivos: demonstrar que a oração é essencial para que tudo isso aconteça e fornecer os meios para fazer da igreja local uma igreja multiplicadora através da oração\cite[p. 29]{brandao}. \end{citacao}

Moisés, por exemplo, falava com Deus face a face, como quem fala a um amigo. Moisés tinha um relacionamento intenso de amizade com Deus que refletia diretamente em seu ministério. Assim entendemos que "a manutenção de uma vida de oração sem cessar é a plenitude do relacionamento entre homem e Deus" \cite[p. 30]{brandao}. Orar é mais do que pedir, pelo contrário, envolve diferentes aspectos:

\begin{itemize}
	\item Reconhecimento: reconhecer quem é Deus e qual é a nossa posição em relação a Ele nos leva a nos prostrarmos diante do Senhor (Lucas 5:8).
	\item Intimidade: Fomos criados para viver em permanente comunhão com Deus, e portanto, precisamos investir o melhor do nosso tempo diário em conversa com Ele, oração e leitura da Palavra.
	\item Santidade: Nossos pecados fazem separação entre nós e Deus, portanto confissão e arrependimento devem fazer parte da oração. Devemos, ao exemplo do salmista Davi, pedir que o Senhor sonde nosso coração e nos leve ao caminho do arrependimento e justiça.
	\item Humildade: Infelizmente em nossos dias, muitos tem dificuldade de ter uma atitude de verdadeira submissão a Deus. Orar é se prostrar diante de Deus em total submissão e reverencia, como um escravo vulnerário de frente a seu Senhor (Lucas 5:8, Filipenses 2:10).
	\item Fé: Precisamos crer que o Senhor ouve as nossas orações, e no seu tempo, de acordo com seus propósitos, nos escuta.
	\item Submissão à vontade de Deus: A oração que aprendemos com Jesus nos ensina a clamar para que a vontade do Pai seja realizada, não a nossa. No Getsêmani jesus ora "Pai, se queres, afasta de mim este cálice. Contudo, que seja feita a tua vontade, e não a minha" (Lucas 22:42).

	\cite[p. 30,31]{brandao}  
\end{itemize}

  A oração é portanto, essencial para o cumprimento da Grande Comissão. A Bíblia tem vários exemplos de homens e mulheres, que viram um agir sobrenatural e extraordinário de Deus como resposta a suas orações. Através da oração, somos capacitados, motivados e orientados pelo Espírito Santo. A partir da oração, os discípulos em Atos 1:8 pregaram com ousadia, e o resultado foram igrejas sendo iniciadas por todas as partes. Precisamos orar para viver de fato a plenitude do Espírito Santo para um testemunho frutífero como o a igreja primitiva \cite[p. 32]{brandao}.

  Da perspectiva do cumprimento da missão de cada discípulo, a oração é de importância estratégica:

  \begin{itemize}
	\item A oração é essencial na evangelização: É preciso investir tempo de oração por conversões, pois somente o Espírito Santo pode convencer as pessoas do pecado. Precisamos preparar o solo, pagando preço da oração para que nossa evangelização seja eficiente.
	\item A oração nos coloca em um papel de parceria com Deus, no processo de libertação dos cativos: \begin{citacao}"O deus deste mundo cegou a mente dos que não creem, para que não consigam ver a luz das boas-novas, não entendendo esta mensagem a respeito da glória de Cristo, que é a imagem de Deus." (2 Cor 4:4).\end{citacao}
	\item Através da intercessão, toda a igreja pode se posicionar estrategicamente na obra missionária. Todos os discípulos podem causar impacto evangelístico por meio da oração: \begin{citacao}"Orem no Espírito em todos os momentos e ocasiões. Permaneçam atentos e sejam persistentes em suas orações por todo o povo santo. E orem também por mim. Peçam que Deus me conceda as palavras certas, para que eu possa explicar corajosamente o segredo revelado pelas boas-novas." (Efésios 6:18,19).\end{citacao}
	\item Jesus nos ensinou a orar pelos vocacionados, um grande desafio para a obra missionária em todo o mundo. Multiplicação de igrejas implica em multiplicação de líderes: \begin{citacao}"Quando viu as multidões, teve compaixão delas, pois estavam confusas e desamparadas, como ovelhas sem pastor. Disse aos discípulos: “A colheita é grande, mas os trabalhadores são poucos. Orem ao Senhor da colheita; peçam que ele envie mais trabalhadores para seus campos"(Mateus 9:36-38).\end{citacao}
	\item A Oração é uma poderosa arma para a batalha espiritual, e elemento indispensável da armadura de Deus: \begin{citacao}"Pois nós não lutamos contra inimigos de carne e sangue, mas contra governantes e autoridades do mundo invisível, contra grandes poderes neste mundo de trevas e contra espíritos malignos nas esferas celestiais. Portanto, vistam toda a armadura de Deus, para que possam resistir ao inimigo no tempo do mal. Então, depois da batalha, vocês continuarão de pé e firmes. "(Efésios 6:12).\end{citacao}
  
	\cite[p. 32-34]{brandao}
\end{itemize}

Os discípulos em trânsito global de hoje se deparam com desafios culturais e sociais da mesma maneira que os discípulos dispersos em Antioquia da Síria. Esses discípulos haviam aprendido a desenvolver um estilo de vida onde a oração era parte fundamental "Todos se dedicavam de coração ao ensino dos apóstolos, à comunhão, ao partir do pão e à oração."(Atos 2:42), o que também colocaram em prática em Antioquia, um processo contínuo e a partir do qual Paulo e Barnabé são enviados: \begin{citacao}"Entre os profetas e mestres da igreja de Antioquia da Síria estavam Barnabé e Simeão, chamado Negro, Lúcio de Cirene, Manaém, que tinha sido criado com o rei Herodes Antipas, e Saulo. Certo dia, enquanto adoravam o Senhor e jejuavam, o Espírito Santo disse: “Separem Barnabé e Saulo para realizarem o trabalho para o qual os chamei”. Então, depois de mais jejuns e orações, impuseram as mãos sobre eles e os enviaram em sua missão."(Atos 13:1-3)\end{citacao}

Conforme destacado anteriormente, a oração é de importância estratégica. Discípulos em trânsito precisam perceber os muitos povos e grupos étnicos ao seu redor, suas necessidades e demandas, carências, conflitos, e sobretudo, o vazio existencial e espiritual, que mantém em estado de morte todo aquele que ainda não confessou Jesus como único Senhor de sua vida. Obedecer a Jesus que nos ensina a orar pelos perdidos, é trabalhar em parceria com o Espírito Santo, o único com poder para conduzi-los para a família de Deus. 

\section{Evangelização Discipuladora}

%\subsubsection{Evangelização}
Evangelizar, é por essência, anunciar o Evangelho, o primeiro passo para a formação de um discípulo de Jesus. A conotação da palavra no Novo Testamento é, entre outros, de "anunciar", "testemunhar", "proclamar"\cite[p. 54]{brandao}. Paulo, em sua carta aos Romanos, exemplifica: \begin{citacao}"Pois “todo aquele que invocar o nome do Senhor será salvo”. Mas como poderão invocá-lo se não crerem nele? E como crerão nele se jamais tiverem ouvido a seu respeito? E como ouvirão a seu respeito se ninguém lhes falar? E como alguém falará se não for enviado? Por isso as Escrituras dizem: “Como são belos os pés dos mensageiros que trazem boas-novas!”. Nem todos, porém, aceitam as boas-novas, pois o profeta Isaías disse: “Senhor, quem creu em nossa mensagem?”. Portanto, a fé vem por ouvir, isto é, por ouvir as boas-novas a respeito de Cristo."(Romanos 10:13-17).\end{citacao}

A evangelização oferece para as pessoas uma oportunidade de receber o Evangelho de maneira clara, de modo que a mesma tenha a possibilidade de responder positivamente. Devemos evangelizar todas as pessoas, "a tempo e fora de tempo" (2 Timóteo 4:2), ou seja, a constante proclamação do Evangelho para as pessoas do nosso alcance. Evangelizar nos une a Jesus em compaixão pelas almas perdidas: "Quando viu as multidões, teve compaixão delas, pois estavam confusas e desamparadas, como ovelhas sem pastor." (Mateus 9:36). A evangelização pode ser portanto, definida como: 

\begin{citacao}A comunicação clara e fiel do Evangelho visando, sob o poder e a dependência do Espírito Santo, levar pessoas ao arrependimento e à fé em Jesus Cristo como Senhor e Salvador, tornando-se seus discípulos \cite[p. 57]{brandao}.\end{citacao}

A evangelização é o primeiro passo para o cumprimento do chamado de cada discípulo. O discipulado segue a evangelização, e começa com um relacionamento intencional desde a primeira relação evangelística. A pessoa evangelizada é então agregada e aperfeiçoada, se tornando um novo discípulo de Jesus. A esse "processo de fazer discípulos multiplicadores por meio do relacionamento intencional de um discípulo com uma pessoa visando torná-la outro discípulo"\cite[p. 64]{brandao}, a Igreja Multiplicadora chama de Relacionamento Discipulador, ou RD. "Enquanto o discipulado é o processo de fazer discípulos, o RD é o meio pelo qual esse processo se desenvolve"\cite[p. 64]{brandao}.

Tanto evangelização quanto discipulado se completam e precisam estar debaixo da autoridade da grande comissão para serem plenamente entendidos e usados pela igreja:

\begin{citacao}O cumprimento da Grande Comissão exige a aplicação conjunta dessas duas forças: a transmissão de verdades por intermédio da comunicação do Evangelho e seus desdobramentos, o que compreende a sua proclamação, exposição e ensino, inclusive quanto a suas implicações para a vida cristã, e a transmissão de	vida por meio de um relacionamento \cite[p. 66]{brandao}.\end{citacao} 

Na descrição do esforço missionário do apóstolo Paulo podemos observar as várias dimensões da grande comissão na maneira como foi realizado seu ministério entre os efésios:

\begin{citacao}"Quando chegaram, ele lhes disse: “Vocês sabem que, desde o dia em que pisei na província da Ásia até agora, fiz o trabalho do Senhor humildemente e com muitas lágrimas. Suportei as provações decorrentes das intrigas dos judeus e jamais deixei de dizer a vocês o que precisavam ouvir, seja publicamente, seja em seus lares. Anunciei uma única mensagem tanto para judeus como para gregos: é necessário que se arrependam, se voltem para Deus e tenham fé em nosso Senhor Jesus. (...) Mas minha vida não vale coisa alguma para mim, a menos que eu a use para completar minha carreira e a missão que me foi confiada pelo Senhor Jesus: dar testemunho das boas-novas da graça de Deus. Por isso, declaro hoje que, se alguém se perder, não será por minha culpa, pois não deixei de anunciar tudo que Deus quer que vocês saibam. (...) Portanto, vigiem! Lembrem-se dos três anos que estive com vocês, de como dia e noite nunca deixei de aconselhar com lágrimas cada um de vocês. (...) Fui exemplo constante de como podemos, com trabalho árduo, ajudar os necessitados, lembrando as palavras do Senhor Jesus: ‘Há bênção maior em dar que em receber" (Atos 20:18-21,24,26,27,31,35)\end{citacao}

Evangelização Discipuladora, ao modelo do ministério de Paulo, precisa incluir transmissão de verdades e de vida, ensino, comunicação e convívio, evangelização e relacionamento. Esses elementos também precisam fazer parte de discípulos em movimento global. Estabelecer intencionalmente relacionamentos significativos precisa fazer parte do seu estilo de vida. É através desses relacionamentos que se pode entender a cultura e perspectiva de vida de pessoas de outras origens, de maneira a transmitir a mensagem do Evangelho contextualizada. As grandes cidades de hoje apresentam um contexto de variação cultural como o de Antioquia da Síria. Pessoas de muitas culturas e religiões, muitos longe da terra natal, buscam desenvolver relacionamentos que muitas vezes são difíceis pela distância que sua cultura apresenta da cultura do local onde estão inseridos. Muitos, com terríveis histórias de sofrimento e fuga de zonas de conflito, chegam a um país de refúgio depois de passar por literalmente muitos países e perder tudo, menos a vida. A percepção dessas realidades precisa sensibilizar discípulos em movimento a agir com intencionalidade, construindo pontes, abrindo espaço para a evangelização, mas também criando uma proximidade de relacionamento que permitirá gerar Cristo em um novo discípulo.

\section{Plantação de Igrejas}

A Grande Comissão nos entrega a missão de fazer discípulos. A igreja é uma comunidade onde discípulos são agregados e aperfeiçoados, e a partir dos quais novos discípulos são formados e o ciclo se repete. O RD como ferramenta para a Evangelização Discipuladora, agrega discípulos a uma igreja local. Nos lugares onde não há igrejas, esse processo implica na plantação de novas igrejas. Dessa maneira, o Evangelho pode ser permanentemente enraizado em uma cidade ou bairro por exemplo, e a partir dali se multiplicar\cite[p. 99,100]{brandao}. Essa foi a estratégia adotada pela igreja primitiva: 

\begin{citacao}"Depois de terem anunciado as boas-novas em Derbe e feito muitos discípulos, Paulo e Barnabé voltaram a Listra, Icônio e Antioquia da Pisídia, onde fortaleceram os discípulos. Eles os encorajaram a permanecer na fé, lembrando-os de que é necessário passar por muitos sofrimentos até entrar no reino de Deus. Paulo e Barnabé também escolheram presbíteros em cada igreja e, com orações e jejuns, os entregaram aos cuidados do Senhor, em quem haviam crido."(Atos 14:21-23).\end{citacao}

Conforme os discípulos foram sendo multiplicados e espalhados pelo mundo, novas comunidades foram sendo criadas. A partir do desenvolvimento e crescimento dessas comunidades, novos discípulos foram sendo enviados a partir dessas comunidades, mas as mesmas também se estabeleceram localmente, impactando as novas gerações:

\begin{citacao}Tudo indica que já no final do primeiro século a igreja percebeu a necessidade da ekklesia – igreja local – para o enraizamento do Evangelho nas cidades, províncias e regiões mais distantes entre os gentios. Isso significa, uma vez mais, que o ato de evangelizar alcança uma pessoa, mas o processo de plantar igrejas	faz com que o Evangelho permaneça para as futuras gerações \cite[p. 101]{brandao}.\end{citacao}

O apóstolo Paulo enfatizava essa necessidade, de não somente evangelizar áreas distantes, mas também de estabelecer ali igrejas locais, que proporcionassem a continuidade do processo de fazer discípulos: \begin{citacao}"Para ele, fazer discípulos não significava apenas a pregação inicial ou mesmo a colheita de alguns frutos, mas o amadurecimento e fortalecimento dos novos convertidos com o intuito de agregá-los e estabelece-los em igrejas locais"\cite[p. 101]{brandao}.\end{citacao}

Plantar igrejas vai além de investir em terrenos e prédios. É investir em pessoas, discípulos multiplicadores que juntos cumprem a missão dada por Jesus, individual e corporativamente. O processo de plantar igrejas, a exemplo das igrejas do novo testamento, pode ser feito de maneira orgânica. Começa nas casas, ambiente seguro e informal onde o Evangelho pode ser anunciado livremente, onde os relacionamentos podem acontecer entre amigos.

Um interessante e inspirador exemplo a respeito dessas possibilidades pode ser observado a partir das migrações modernas de Filipinos pelo mundo. Oficialmente, aproximadamente 10\% da população das Filipinas (quase 8 milhões de pessoas) está espalhada por mais de 182 países do mundo. Muitos desses migrantes profissionais estão inseridos em locais onde o Evangelho não é bem-vindo \cite[p. 3]{wan2009filipino}. Eles servem em seus países de destino como médicos, engenheiros, professores, arquitetos, além de outras profissiões. Sua cultura amigável e alegre os permite facilmente se adaptar e envolver com os povos que os recebem, sendo que muitas igrejas e agências missionárias foram iniciadas a partir de grupos de filipinos em diáspora \cite[5,6]{wan2009filipino}. Em Chipre, por exemplo, pode-se encontrar centenas de filipinos reunidos no parque de Nicosia aos domingos de manhã, onde dezenas de grupos de estudo bíblico e grupos de oração acontecem por várias horas. Além disso, filipinos iniciaram muitas igrejas locais, inclusive igrejas clandestinas, em países do Oriente Médio e África. Um caso conhecido é de um grupo de filipinos que reside em um país hostil ao Evangelho, e que portanto começou a se reunir em um ônibus alugado. Aproximadamente 50 pessoas formavam uma igreja, discípulos de Jesus que se encontravam com colegas de trabalho e outros expatriados para adorar a Deus. A igreja no ônibus, iniciada em 2005, cresceu e se multiplicou, sendo que em 2009, já haviam 3 igrejas desse tipo. \cite[8]{wan2009filipino}.

O povo brasileiro é um povo internacionalmente conhecido como um povo amigável, de pessoas alegres e descomplicadas, facilmente adptáveis, simpáticas, nesse aspecto similares aos filipinos. A exemplo dos discípulos Filipinos em trânsito global, discípulos profissionais entre as nações tem à sua disposição muitas oportunidades para o estabelecimento de novas igrejas, de maneiras inovadoras. 

\section{Formação de Líderes}

O maior exemplo de liderança multiplicadora é o próprio Jesus. A parte principal de seu trabalho foi de formar seus líderes, ensinando de maneira prática os valores do Reino de Deus, no desenrolar da vida. O apóstolo Paulo nos relembra que recebemos dons para um propósito definido, o aperfeiçoamento dos santos para a obra do ministério: 

\begin{citacao}"Ele designou alguns para apóstolos, outros para profetas, outros para evangelistas, outros para pastores e mestres. Eles são responsáveis por preparar o povo santo para realizar sua obra e edificar o corpo de Cristo, até que todos alcancemos a unidade que a fé e o conhecimento do Filho de Deus produzem e amadureçamos, chegando à completa medida da estatura de Cristo (Efésios 4:11-13)."\end{citacao}

Também percebemos no texto de Paulo, que essa edificação deve ocorrer em outros e em nós mesmos também, ou seja, o chamado para exercer a liderança ministerial é também um chamado para o aperfeiçoamento da própria vida cristã. O líder não é forte em si mesmo, mas dependente da força que vem do próprio Mestre para seguir no Caminho\cite[p. 130,131]{brandao}. A liderança traz em si inúmeros desafios, que precisam ser enfrentados e superados em Cristo. O líder precisa ser, em cada experiência, preparado, e entender a importância de estar revestido do poder de Deus e capacitado a utilizar as armas espirituais em sua trajetória de liderança. \cite[p. 132]{brandao}.

Uma das tarefas mais importantes do líder é preparar os santos para a continuidade da obra missionária, através dos dons e talentos recebidos, e para a glória de Deus, de modo que o Evangelho chegue até os confins da terra. Além disso, a preparação de líderes habilita o crescimento da igreja. O conselho de Jetro a Moisés relatado em Êxodo capítulo 18 nos mostra a importância da formação de líderes para que o ministério possa ganhar escala: 

\begin{citacao}"Agora ouça-me e escute meu conselho, e Deus esteja com você. Continue a ser o representante do povo diante de Deus, apresentando-lhe as questões trazidas pelo povo. Ensine a eles os decretos e as instruções de Deus. Mostre aos israelitas como devem viver e o que devem fazer. No entanto, escolha dentre todo o povo homens capazes e honestos que temam a Deus e odeiem suborno. Nomeie-os líderes de grupos de mil, cem, cinquenta e dez pessoas. Eles deverão estar sempre disponíveis para resolver os problemas cotidianos do povo e só lhe trarão os casos mais difíceis. Deixe que os líderes decidam as questões mais simples por conta própria. Eles dividirão com você o peso da responsabilidade e facilitarão seu trabalho. Se você seguir esse conselho, e se Deus assim lhe ordenar, poderá suportar as pressões, e todo este povo voltará para casa em paz." (Êxodo 18:19-23).\end{citacao}

Nas relações discipulares do Antigo e Novo Testamento, fica claro a formação de linhagens discipulares:  \begin{citacao}"Moisés preparou Josué; assim como Elias preparou Eliseu; Jesus preparou seus apóstolos; Barnabé preparou Paulo, que preparou Silas, Timóteo, Tito, e assim por diante. Precisamos, da mesma forma, preparar líderes para que nossas igrejas não sofram no processo de crescimento e não deixem de se multiplicar."\cite[p. 137]{brandao}.\end{citacao}

A partir da igreja de Antioquia, vemos mais uma vez a intencionalidade de um líder no processo de formação de outro líder: \begin{citacao}"Barnabé era um homem bom, cheio do Espírito Santo e de fé. E uma grande multidão se converteu ao Senhor. Então Barnabé foi a Tarso procurar Saulo. Quando o encontrou, levou-o para Antioquia. Ali permaneceram com a igreja um ano inteiro, ensinando a muitas pessoas. Foi em Antioquia que os discípulos foram chamados de cristãos pela primeira vez." (Atos 11:24-26)\end{citacao}.

Barnabé nos dá o exemplo de como formar um líder estratégico (Paulo) para o Reino de Deus, alguém que foi primeiramente cuidado, encorajado, preparado, e afinal, enviado para produzir muito fruto e gerar outros discípulos.

\section{Compaixão e Graça}

As primeiras igrejas cristãs se tornaram relevantes através de ações de compaixão e graça em suas comunidades locais. À oração, testemunho público, e anunciação do Evangelho, se somou uma atitude de compaixão que vinha de encontro às pessoas e suas necessidades espirituais, físicas e emocionais. As igrejas eram assim percebidas, posicionadas como elemento de transformação na sociedade\cite[p. 165]{brandao}. Entre as várias narrativas de ações de compaixão, a história da primeira comunidade de discípulos relatada em Atos dos Apóstolos é muito significativa: 

\begin{citacao}"Todos se dedicavam de coração ao ensino dos apóstolos, à comunhão, ao partir do pão e à oração. Havia em todos eles um profundo temor, e os apóstolos realizavam muitos sinais e maravilhas. Os que criam se reuniam num só lugar e compartilhavam tudo que possuíam. Vendiam propriedades e bens e repartiam o dinheiro com os necessitados, adoravam juntos no templo diariamente, reuniam-se nos lares para comer e partiam o pão com grande alegria e generosidade, sempre louvando a Deus e desfrutando a simpatia de todo o povo. E, a cada dia, o Senhor lhes acrescentava aqueles que iam sendo salvos." (Atos 2:42-27)\end{citacao}

A mesma atitude precisa ser uma marca da igreja de nossos dias, o amor, cuidado mútuo, e zelo entre os discípulos de hoje precisam ser tão marcantes quanto no princípio, de modo que muitos de fora sejam atraídos para a comunidade, a família de Deus. Nas palavras de Jesus: "Seu amor uns pelos outros provará ao mundo que são meus discípulos." (João 13:35). Muitos de fora foram atraídos para essa nova comunidade ao ver a unidade de coração e propósito que havia entre eles:

\begin{citacao}"Os apóstolos realizavam muitos sinais e maravilhas entre o povo. Todos se reuniam regularmente no templo, na parte conhecida como Pórtico de Salomão. Quando se reuniam ali, ninguém mais tinha coragem de juntar-se a eles, embora o povo os tivesse em alta consideração. Cada vez mais pessoas, multidões de homens e mulheres, criam no Senhor. Como resultado, o povo levava os doentes às ruas em camas e macas para que a sombra de Pedro cobrisse alguns deles enquanto ele passava. Muita gente vinha das cidades ao redor de Jerusalém, trazendo doentes e atormentados por espíritos impuros, e todos eram curados." (Atos 5:12-16)\end{citacao}

Podemos observar que as primeiras igrejas tinham muita clareza sobre sua identidade e missão. Suas obras de compaixão e justiça agregaram muitos, e impactaram a sociedade em que estavam inseridas. Nós também, como corpo de Cristo devemos estender as mãos, abrir os olhos e o coração para as necessidades dos que estão ao nosso redor. Assim, nos identificaremos com Jesus, nosso maior exemplo de compaixão e graça. Jesus, depois de um dia de serviço, se compadeceu de uma multidão que não tinha o que comer, vindo de encontro a suas necessidades (Marcos 6:34-44), cuidou de dois endemoninhados gadarenos após uma tempestade no mar da Galileia (Mateus 8:28-34), entre tantas outras demonstrações de compaixão. A igreja de Cristo precisa ser sensível com a situação das pessoas à sua volta; chorar, sentir, se envolver\cite[p. 172]{brandao}.

Os discípulos em movimento global, inspirados pelo exemplo de Jesus, tem muitas oportunidades de demonstrar compaixão e justiça, de exercer o serviço cristão. Oportunidades de testemunho surgem durante o serviço na comunidade: É possível participar da equipe de pais que vai fazer a limpeza das salas de aula da escola dos filhos, comprar comida em um restaurante que pertence a refugiados de maneira a deliberadamente ajudá-los, acolher em sua casa um migrante solitário que se conheceu durante as compras no super mercado, entre tantos outros. Assim como os discípulos de Antioquia que abriram caminho para o evangelho entre os gentios, é necessário um coração sensível, olhos abertos, ouvidos atentos para perceber e explorar as oportunidades ao nosso redor. Além disso, é preciso uma mente aberta e criativa para se aproveitar bem cada situação, em tempo e fora de tempo, na igreja, em casa, no parque, na café dentro da estação de trem. Da mesma maneira que os primeiros discípulos não se deixaram intimdar por barreiras físicas, culturais ou sociais, também através de nós o Evangelho pode ser pregado a todas as pessoas, até os confis da terra, um discípulo de cada vez.

\chapter*{Considerações Finais}
\addcontentsline{toc}{chapter}{CONSIDERAÇÕES FINAIS}  

O movimento iniciado por Jesus Cristo levou discípulos dispersos em Antioquia da Síria a serem testemunhas e, cumprindo seu chamado, fazer discípulos multiplicadores, iniciando a pregação do Evangelho entre os gentios. Esse grupo de discípulos leigos serviu de plataforma para o trabalho missionário do apóstolo Paulo. 

Os movimentos migratórios com o da história de Antioquia da Síria, continuam a acontecer nos tempos modernos, envolvendo um percentual significativo da população mundial, em trânsito. Dentre os diversos fatores que motivam movimentos migratórios, observamos o movimento de profissionais qualificados, em alta demanda em todas as partes, especialmente na Europa e Estados Unidos. Na União Europeia, é possível perceber essa realidade, entre outros, pelo crescente número de "Blue Cards" oferecidos pelos países do bloco entre os anos de 2013 a 2018. Existe, portanto, um grande potencial para um movimento evangelístico e de plantação de igrejas através de discípulos de Jesus com habilidades profissionais específicas. Esses discípulos, aos moldes daqueles que plantaram a igreja de Antioquia da Síria, podem, nos dias de hoje, serem instrumentos para evangelismo e discipulado em locais e contextos (sociais e profissionais) onde iniciativas missionárias organizadas teriam dificuldade de chegar.

Discípulos comprometidos com Jesus Cristo podem intencionalmente desenvolver competências profissionais que os habilitem a ser uma força evangelística de impacto global. O trabalho conclui mostrando que é possível ser um discipulador, plantador de igrejas e cumprir o mandado da Grande comissão a partir do desenvolvimento de competências profissionais específicas, em demanda na atualidade. Discípulos leigos podem se somar a missionários no processo de ir até os confins da terra.

\bibliography{research}

\end{document}
