\documentclass[
	% -- opções da classe memoir --
	12pt,				% tamanho da fonte
	openright,			% capítulos começam em pág ímpar (insere página vazia caso preciso)
	twoside,			% para impressão em recto e verso. Oposto a oneside
	a4paper,			% tamanho do papel. 
	% -- opções da classe abntex2 --
	%chapter=TITLE,		% títulos de capítulos convertidos em letras maiúsculas
	%section=TITLE,		% títulos de seções convertidos em letras maiúsculas
	%subsection=TITLE,	% títulos de subseções convertidos em letras maiúsculas
	%subsubsection=TITLE,% títulos de subsubseções convertidos em letras maiúsculas
	% -- opções do pacote babel --
	english,			% idioma adicional para hifenização
	french,				% idioma adicional para hifenização
	spanish,			% idioma adicional para hifenização
	brazil				% o último idioma é o principal do documento
	]{abntex2}

% ---
% Pacotes básicos 
% ---
\usepackage{lmodern}			% Usa a fonte Latin Modern			
\usepackage[T1]{fontenc}		% Selecao de codigos de fonte.
\usepackage[utf8]{inputenc}		% Codificacao do documento (conversão automática dos acentos)
\usepackage{indentfirst}		% Indenta o primeiro parágrafo de cada seção.
\usepackage{color}				% Controle das cores
\usepackage{graphicx}			% Inclusão de gráficos
\usepackage{microtype} 			% para melhorias de justificação
% ---
		
% ---
% Pacotes adicionais, usados apenas no âmbito do Modelo Canônico do abnteX2
% ---
\usepackage{lipsum}				% para geração de dummy text
% ---

% ---
% Pacotes de citações
% ---
\usepackage[brazilian,hyperpageref]{backref}	 % Paginas com as citações na bibl
\usepackage[alf]{abntex2cite}	% Citações padrão ABNT

\title{Princípios da Igreja Multiplicadora aplicados na vida de proifssionais em movimento global para abertura de frentes evangelizadores e/ou fortalecimento de igrejas.}
%TODO Pensar em como reduzir o título
%Proposta de novo Objetivo Geral(): Estuda de que maneiras os princípos da igreja multiplicadora podem apoisar os cristãos leigos em trânsito no cumprimento da Grande Comissão para abrir e fortalecer Igrejas
%TODO Mudar Objetivo Geral
\author{Gabriel Kalil}
\date{ }
  
\begin{document}
  
\maketitle
  
\tableofcontents

\chapter{Introdução}


\chapter{O surgimento da Igreja e os discípulos em movimento evangelizador}

% O conceito de movimento e espontaniedade é que tem que aparecer aqui. Essa visão é que tem que estar aqui.
% Tem que virar um artigo que igrejas e pessoas podem usar. Se Deus me mandar para Pequim, é uma missão estar lá.
% Christopher Wright (a missão de Deus), diz que Deus é um Deus em movimento (ver pespectivas), como Jesus ele põe o povo em movimento.

\section{O surgimento da Igreja}
   
A primeira vez que a palavra "igreja" aparece no Novo Testamento é na conversa entre Jesus e Pedro onde o primeiro afirma ser Jesus o "Cristo, o Filho do Deus vivo" ao que Jesus responde "Agora eu lhe digo que você é Pedro, e sobre esta pedra edificarei minha igreja (Mt 16.18). Apesar de várias hipóteses e argumentos a respeito do real significado da "pedra", outras passagens do Novo Testamento defendem a idéia de que o principal protagonista na fundação da Igreja é o próprio Jesus Cristo, como argumenta Paulo na carta aos Coríntios
"pois ninguém pode lançar outro alicerce além daquele que já foi posto, isto é, Jesus Cristo." (1 Coríntios 3:11) bem como aos Efésios (Ef 2.20,21). 

%Pedra, não entrar nesta parte, não provocar a tese que eu não posso trabalhar.
%Trabalhar o surgimento da Igreja e o que significa ser Igreja, deixar o conteúdo mais técnico. Igreja surge entre os discípulos, surge como comunidade entre os primeiros discípulos, TCC Batista Pioneira (Mariane). No meu caso, como vou falar em movimento, precisamos colocar o fato que a igreja nasce no movimento de discípulos de Jesus. 
%Definir Igreja Universal e Igreja Local. Lá na frente, vamos precisar desta definição. Nasce em pequenos grupos, num aspecto de Kononia.

%Ver estrutura que o Marcos colocou no documento. Talvez citar Atos 2, Igreja em movimento já voltando para o local de origem (Pode entrar aí). Citar Atos 2. Pessoas de várias nacionalidades

A narrativa do livro de Atos dos Apóstolos, já no início menciona a promessa feita por Jesus sobre o evento que marcaria com capacitação sobrenatural, o início do cumprimento da Grande Comissão dada em Mateus 28
\begin{citacao}
Jesus se aproximou deles e disse: "Toda a autoridade no céu e na terra me foi dada. Portanto, vão e façam discípulos de todas as nações, batizando-os em nome do Pai, do Filho e do Espírito Santo. Ensinem esses novos discípulos a obedecerem a todas as ordens que eu lhes dei. E lembrem-se disto: estou sempre com vocês, até o fim dos tempos(At 1.18-20)
\end{citacao}

TODO citar aqui abaixo que aqui já tem pessoas de várias nacionalidades (Atos2)

O cumprimento da promessa no dia de Pentecostes e a subsequente formação da primeira comunidade dos que pertenciam ao "Caminho" se confirmava, pois "a cada dia, o Senhor lhes acrescentava aqueles que iam sendo salvos" (At 2.47). John Stott reconhecendo a característica evangelística dessa igreja, afirma que o crescimento diário não foi um evento esporádico. A igreja embora ainda não organizasse programas de missões, na mesma medida em que adoravam ao Senhor diariamente (At 2.46), também exercitava o seu testemunho. Tanto a adoração quanto a proclamação fluem naturalmente de corações cheios do Espírito Santo. \cite[118-119]{stott}

TODO cuidar com palavras, ao invés de missionário "evangelístico", mais simples porque já define quando falar Mateus 28.
%Cuidar com conceitos que não apareceram. Ver 1.1.4 do Marcos, estilo de vida relacional (explorar mais), nas casas. Não aprofundar tanto mas trazer isso rápido, etc. Prepara o caminho para o profissional que vai receber amigos, já lança a semente pra frente (base teológica) GREEN 262,252

TODO nasce na espontanedade, nas casa, no pátio do templo, livre, movimento. Base para usar profissional depois. Precisar ser igreja movimento.

Esse DNA missionário se manifesta também após Pedro ser instrumento de cura milagrosa de um aleijado, provocando ameaças do conselho de líderes. Assim que foram libertos, Pedro e João retornam para o lugar onde estavam os outros irmãos que ao se identificarem com Pedro e João, juntos levantaram a voz em clamor a Deus. Em resposta a esse clamor e capacitados pelo Espírito Santo, continuaram a pregar as boas novas corajosamente (At 4.21-31).

Observamos portanto que começa a se formar nessa comunidade um sentido de missão universal: a grande comissão de Mateus 28 foi iniciada e todos os santos estavam engajados.

\section{A primeira comunidade cristã e seu foco em Jesus como Senhor}

Com o crescimento da jovem igreja vieram também crescentes problemas. Ao martírio de Estêvão se seguiu uma grande onda de perseguição que dispersou a igreja de Jerusalém para a Judéia, Samaria e além. (At 8.1).

Do ponto de vista cultural, a grande comissão começou a partir da evangelização dos judeus. Esse quadro começa a se modificar quando alguns discípulos fugindo da perseguição capitaneada pelos líderes judeus, chegaram a Antioquia da Síria e lá começaram a pregar as boas novas a pessoas de fala grega, o que segundo Stott se refere tanto a judeus de fala grega quanto a gentios, ou seja, pessoas de várias origens étnicas que habitavam a cidade na época \cite[184]{stott}.

TODO não necessário, não vai usar isso. Talvez precise usar isso lá no profissional. Teria que justificar essa construção pra explicar isso, talvez a contextualização é um aspecto mais missional. Encarnacional, conceito de encarnação. Aculturamento (ver que vale a pena abrir esse leque)

TODO princípio da Igreja multiplicadora. Oração, formação de líderes, compaixão e graça, etc.. Tem que amarrar tudo. Colocar os 5 princípios na frente do computador. ORAÇÃO, EVANGELIZAÇÃO DISCIPULADORA (INFLUENCIA O 1 CAPÍTULO), plantação de igrejas (mentalidade), formação de líderes, compaixão e graça (essa parte de contextualização vem aqui). 

%No capítulo 1, teria que trazer Atos 2:42 (mais do que Pedro). Tem 4 princípios, Koinonia (viviam em comunhão), Diaconia (compaixão e graça) => lá na frente, o profissional precisa estar atento a quais as necessidasdes daquele povo (empatia, simpatia, como me relaciono nesse povo), generosidade (partilha do pão), didake (ensino) => ensino fundamentado, doutrina dos apóstolos.
TODO posso gastar algumas linhas para colocar o estilo de vida deles, vai dar a base para os princípios de igreja multiplicadora. Didake (evangelização discipuladora, que ensina). Oração (ver doc MARCOS). Koinonia (trabalharia na frente pequenos grupos), Generosidade (também compaixão e graça). Faltando aqui a formação de líderes (dar um toque nisso).

TODO Preparar a terrar para o que vamos falar lá na frente. Senão lá vamos puxar algo novo que nunca foi falado.

Nese contexto multi cultural, inovações na proclamação da boa nova começaram a ser introduzidas: Jesus era apresentado não somente como o Cristo (identificação com o Messias esperado pelos judeus) mas agora também, e com mais ênfase, como "O Senhor". Essa inovação terminológica e conceitual, tinha primariamente um caráter apologético. No mundo helênico de muitos deuses e senhores, apresentava Jesus como o único Deus e único Senhor. Mais que isso, chamava atenção para a soberania de Jesus sobre todas as forças de destino que ameaçavam as pessoas. \cite[170]{green}. Jesus, o Senhor, tinha autoridade sobre todas as forças, inclusive sobre a morte (Rm 8.38).

\section{Antioquia da Síria, a base de lançamento para missões gentílicas.}
%Aqui é isso mesmo, não tem muito o que mexer

A base de lançamento para o projeto missionário entre os gentios foi a cidade de Antioquia da Síria, conforme (Atos 11:19-26,13:1-3), onde Lucas relata pela primeira vez a pregação do Evangelho para pessoas totalmente pagãs, uma mudança decisiva na história da igreja. A inovação foi aceita pela igreja de Jerusalém e confirmada pelo envio de Barnabé. Este fora comissionado para aprovar e comunicar à nova comunidade o entendimento de Jerusalém de que os gentios convertidos não precisavam cumprir os ritos da Lei de Israel, inclusive a circuncisão. A fé e o batismo eram suficientes para identificá-los com Jesus, independentemente de sua origem étnica. \cite[198]{green}.


Antioquia era a capital da província da Síria, uma cidade extremamente cosmopolita. Apesar de ser de fundação grega, seus estimados 500.000 habitantes tinham origens variadas. A uma grande colônia de Judeus, se somavam habitantes oriundos da Pérsia, Índia, China, latinos. Gregos, judeus, orientais e romanos formavam uma combinação que levou o historiador Josephus a considerá-la a terceira cidade do império, atrás somente de Roma e Alexandria \cite[185]{stott}. A cidade apresentava um ambiente interessante para o alcançe dos gentios uma vez que as barreiras entre os judeus e eles eram muito tênues, haviam muitos convertidos ao judaísmo e os judeus tinham direitos de cidadãos. Por ser um dos grandes centros comerciais e uma das maiores cidades do império, desenvolvera relações comerciais por todo o mundo o que facilitava o trânsito de pessoas num contexto multi nacional e multi cultural, um literal ponto de convergência entre o Ocidente e o Oriente. Vários sistems de crença também se encontravam ali, desde o culto a Zeus e outros deuses gregos, Baal (de origem síria), religiões de mistério, entre outros.\cite[166,167]{green}. Com um contexto cultural e religioso tão variado, Antioquia da Síria poderia ser facilmente  confundida com capitais Européias como Londres ou Berlin de nossos tempos.

\section{Fundada por missionários informais}
TODO retirar, não precisa porque já colocamos todos os elementos anteriormente. Até aqui, dar um panorama muito bom que é a igreja movimento. Aqui fechar, mostrando o exemplo de uma igreja que nasce deste movimento.

Falar sobre características interessantes que possam ter a ver com o estudo
(o pessoal foi parar em Antioquia mais o dna missionário estava dentro deles. A cidade e contexto deram oportunidade para a continuidade do trabalho)

TODO guardar uma pérola para depois, outro momento.TALVEZ para o livro, ou algo assim. As conversas começam apertando este calo.
\section{Base fraca para valores}

Antioquia da Síria, ao sofrer influências gregas e romanas também tinham um sistema de crença em deuses que não eram mais do que "super humanos", não divindades. A falta de um deus infinito levava a uma dificuldade intelectual, a falta de um ponto de referência absoluto. Não havia nada imutável a partir do qual construir a vida. Seu sistema de valores, consequentemente não dava uma base forte para a construção de uma vida que fazia sentido, nem ao se combinar todos os seus deuses juntos. Esses deuses eram produto e dependentes da sociedade \cite[19]{schaeffer}. A falta de referência era tanta que o povo romano aceitou a idéia de abandonar a república e aceitar um governo ditatorial de César Augusto. Segundo Plutarco, "os romanos colocaram César como ditador vita'licio na esperança de que o governo de uma pessoa única pudesse lhes dar tempo para respirar após tantas guerras civis e calamidades. Essa foi de fato, uma tirania autorizada, uma vez que o poder era agora não somente absoluto mas também perpétuo" \cite[19]{schaeffer}.

Após o ano 12 BC (CE)?, Augusto recebeu o título de Pontifex Maximus, líder da religião do estado, conclamando o povo à adoração ao espírito de Roma e ao imperador, o que dentro de pouco tempo se tornou obrigatório e levou os imperadores romanos a serem venerados como deuses. A decadência espiritual e intelectual, levou também a uma decadência moral, sendo Antioquia criticada por autores seculares como Juvenal por sua fama de imoral. \cite[166]{green}. Os imperadores tentaram novamente dar sentido para seus liderados. Augusto, por exemplo, se ocupou em atividades para definir valores morais e de vída familiar. Um deus humano, no entanto ainda, é finito, uma fundação insufienciente para uma vida de sentido e valor. \cite[20]{schaeffer}. 

O cenário espiritual e intelectual do império romano clamava por alguém maior do que a vida, e que pudesse dar para a vida o sentido e valor que não se encontrava em deuses super humanos ou em deuses humanos. A mensagem de um deus eterno, absoluto, criador e doador de vida, sentido e propósito, imanente e presente, não poderia ter deixado de causar um tremendo impacto. Chegou na hora certa, ou nas palavras do apóstolo Paulo, "na plenitude dos tempos" (GALATAS 4:4).

\bibliography{research}

\end{document}